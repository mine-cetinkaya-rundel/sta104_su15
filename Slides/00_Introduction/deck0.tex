% -*- TeX-engine: xetex; eval: (auto-fill-mode 0); eval: (visual-line-mode 1); -*-
% Compile with XeLaTeX

%%%%%%%%%%%%%%%%%%%%%%%
% To do before class
%%%%%%%%%%%%%%%%%%%%%%%

% Send the Logistics/Week0Annoucnement (the night before).
% Send an email reminding students to bring a charged computer (the night before).

%%%%%%%%%%%%%%%%%%%%%%%
% Option 1: Slides: (comment for handouts)   %
%%%%%%%%%%%%%%%%%%%%%%%

\documentclass[slidestop,compress,mathserif,12pt,t,professionalfonts,xcolor=table]{beamer}

% solution stuff
\newcommand{\solnMult}[1]{
\only<1>{#1}
\only<2->{\red{\textbf{#1}}}
}
\newcommand{\soln}[1]{\textit{#1}}

%%%%%%%%%%%%%%%%%%%%%%%%%%%%%%%
% Option 2: Handouts, without solutions (post before class)    %
%%%%%%%%%%%%%%%%%%%%%%%%%%%%%%%

% \documentclass[11pt,containsverbatim,handout,xcolor=xelatex,dvipsnames,table]{beamer}

% % handout layout
% \usepackage{pgfpages}
% \pgfpagesuselayout{4 on 1}[letterpaper,landscape,border shrink=5mm]

% % solution stuff
% \newcommand{\solnMult}[1]{#1}
% \newcommand{\soln}[1]{}

% % % This breaks things for me for some reason.
% % tell pgfpages how to set page sizes in XeLaTeX
% %\renewcommand\pgfsetupphysicalpagesizes{%
% %   \pdfpagewidth\pgfphysicalwidth\pdfpageheight\pgfphysicalheight%
% %}

%%%%%%%%%%%%%%%%%%%%%%%%%%%%%%%%%%%%
% Option 3: Handouts, with solutions (may post after class if need be)    %
%%%%%%%%%%%%%%%%%%%%%%%%%%%%%%%%%%%%

% \documentclass[11pt,containsverbatim,handout,xcolor=xelatex,dvipsnames,table]{beamer}

% % handout layout
% \usepackage{pgfpages}
% \pgfpagesuselayout{4 on 1}[letterpaper,landscape,border shrink=5mm]

% % solution stuff
% \newcommand{\solnMult}[1]{\red{\textbf{#1}}}
% \newcommand{\soln}[1]{\textit{#1}}

% % % This breaks things for me for some reason.
% % % tell pgfpages how to set page sizes in XeLaTeX
% % \renewcommand\pgfsetupphysicalpagesizes{%
% %    \pdfpagewidth\pgfphysicalwidth\pdfpageheight\pgfphysicalheight%
% % }

%%%%%%%%%%%%%%%%%%%%%%%%%%%%%%%
% Option 4: Notes Only
%%%%%%%%%%%%%%%%%%%%%%%%%%%%%%%

% % See http://tex.stackexchange.com/questions/114219/add-notes-to-latex-beamer
% \documentclass[10pt,containsverbatim,xcolor=xelatex,dvipsnames,table,notes=only]{beamer}

% % handout layout
% % \usepackage{pgfpages}
% % \pgfpagesuselayout{1 on 1}[letterpaper, landscape, border shrink=5mm]

% % solution stuff
% \newcommand{\solnMult}[1]{#1}
% \newcommand{\soln}[1]{}

% % % Having a problem with this.
% % tell pgfpages how to set page sizes in XeLaTeX
% % \renewcommand\pgfsetupphysicalpagesizes{%
% %   \pdfpagewidth\pgfphysicalwidth\pdfpageheight\pgfphysicalheight%
% %}

%%%%%%%%%%
% Load style file, defaults  %
%%%%%%%%%%

\input{../lec_style.tex}
% You cannot use numbers when defining variables.  Hence the use of letters, A, B, C, etc.

% Personal Info
\newcommand{\FirstName}{Mine}
\newcommand{\LastName}{\c{C}etinkaya-Rundel}
\newcommand{\OfficeHours}{Generally TR 12:30 - 1:30pm}

% Electronic Info
\newcommand{\PersonalSite}{http://stat.duke.edu/~mc301}
\newcommand{\CourseSite}{http://bit.ly/sta104su15}
\newcommand{\Email}{mine@stat.duke.edu}

% TAs
\newcommand{\TAA}{Andrew Wong}

% Exam Dates
\newcommand{\ExamDate}{May 29, 11am - 12:30pm (in class)}
\newcommand{\FinalDate}{June 24, 11am - 2pm}

% ALT ALT
% \input{../definitions_custom.tex}

%%%%%%%%%%%
% Cover slide info    %
%%%%%%%%%%%

\title{Data Analysis and Statistical Inference}
\subtitle{Introduction}
\author{Sta 104 - Summer 2015}
\date{May 13, 2015}
% ALT ALT
% \date{January 8, 2015}
\institute{Duke University, Department of Statistical Science}


%%%%%%%%%%%%%%%%%%%%%%%%%
% Begin document and set Helvetica Neue font   %
%%%%%%%%%%%%%%%%%%%%%%%%%

\begin{document}
\fontspec[Ligatures=TeX]{Helvetica Neue Light}

%%%%%%%%%%%%%%%%%%%%%%%%%%%%%%%%%%%

% Title Page

\begin{frame}[plain]

\titlepage
\vfill
{\scriptsize \webLink{\PersonalSite}{Dr. \LastName{}} \hfill Slides posted at  \webLink{\CourseSite}{\CourseSite}}
\addtocounter{framenumber}{-1} 

\end{frame}

%%%%%%%%%%%%%%%%%%%%%%%%%%%%%%%%%%%

\section{Introductions}

%%%%%%%%%%%%%%%%%%%%%%%%%%%%%%%%%%%

\begin{frame}
\frametitle{Introductions}

\begin{itemize}

\item Teaching team:
\begin{itemize}
\item Professor: Dr. Mine \c{C}etinkaya-Rundel
\item TA: Andrew Wong
\end{itemize}

\item Student teams:
\begin{itemize}
\item Team 1: Alex, Logan, Kaia 	
\item Team 2: Matthew, Jaidev, Melissa 	
\item Team 3: Kathleen, Frannie, Albert 
\item Team 4: Abhi, Max, Samhita, Caitlin 	
\end{itemize}

\end{itemize}

\end{frame}

%%%%%%%%%%%%%%%%%%%%%%%%%%%%%%%%%%%

\section{General info}

%%%%%%%%%%%%%%%%%%%%%%%%%%%%%%%%%%%

\begin{frame}
\frametitle{Required materials}

\begin{itemize}

\item OpenIntro Statistics, 2nd Edition: \webURL{https://www.openintro.org/}

\item (optional) Calculator

\end{itemize}


\end{frame}

%%%%%%%%%%%%%%%%%%%%%%%%%%%%%%%%%%%

\begin{frame}
\frametitle{Webpage}

\vfill

\centering
{\Large 
\webURL{\CourseSite}
}

\vfill

\end{frame}

%%%%%%%%%%%%%%%%%%%%%%%%%%%%%%%%%%%

\begin{frame}
\frametitle{Grading}

\begin{center}
\rowcolors{1}{}{custom_lightGray}
\renewcommand\arraystretch{1.25}
{\scriptsize
\begin{tabular}{ r | l }
\textbf{Component} & \textbf{Weight} \\
Attendance \& participation + peer evaluation	& 10\% \\
Problem sets							& 10\%  \\ 
Labs									& 10\% \\    
Readiness assessments					& 10\%   \\  
Performance assessments 				& 5\%  \\  
Project 1								& 10\% \\   
Midterm 1 							& 20\% \\    
Final 								& 25\%     
\end{tabular}
}
\end{center}

\begin{itemize}

\item Grades may be curved at the end of the semester. 

\item Cumulative numerical averages of 90 - 100 are guaranteed at least an A-, 80 - 89 at least a B-, and 70 - 79 at least a C-, however the exact ranges for letter grades will be determined after the final exam. 

\item The more evidence there is that the class has mastered the material, the more generous the curve will be.

\end{itemize}

\end{frame}

%%%%%%%%%%%%%%%%%%%%%%%%%%%%%%%%%%%

\section{Goals}

%%%%%%%%%%%%%%%%%%%%%%%%%%%%%%%%%%%

\begin{frame}
\frametitle{Course goals and objectives}

{\footnotesize
\begin{itemize}[<alert@+>]
\item Recognize the importance of data collection, identify limitations in data collection methods, and determine how they affect the scope of inference.
\item Use statistical software to summarize data numerically and visually, and to perform data analysis.
\item Have a conceptual understanding of the unified nature of statistical inference.
\item Apply estimation and testing methods to analyze single variables or the relationship between two variables in order to understand natural phenomena and make data-based decisions.
\item Model numerical response variables using a single or multiple explanatory variables.
\item Interpret results correctly, effectively, and in context without relying on statistical jargon.
\item Critique data-based claims and evaluate data-based decisions.
\item Complete two research projects: one that focuses on statistical inference and one that focuses on modeling. 
\end{itemize}
}

\end{frame}

%%%%%%%%%%%%%%%%%%%%%%%%%%%%%%%%%%%

\section{Course structure and components}

%%%%%%%%%%%%%%%%%%%%%%%%%%%%%%%%%%%

\begin{frame}
\frametitle{Course structure}

\begin{itemize}[<alert@+>]
\item Set of learning objectives and required and suggested readings, videos, etc. for each unit
\item Prior to beginning the unit, watch the videos and/or complete the readings and familiarize yourselves with the learning objectives
\item Begin a new unit with a readiness assessment: individual, then team 
\item Class time: split between lecture, discussion/application, and lab
\item Complement your learning with problem sets
\item Wrap up a unit with a performance assessment
\end{itemize}

\end{frame}

%%%%%%%%%%%%%%%%%%%%%%%%%%%%%%%%%%%

\begin{frame}
\frametitle{Teams}

\begin{itemize}
\item Highly functional teams of learners based on survey and pre-test

\item Team members first point of contact

\item Application exercises, labs, team readiness assessments, projects

\item Seek help from your team members, but first attempt individual assignments before asking for help

\item Anything that is not explicitly a team assignment must be your own work

\item Peer evaluations to ensure that all team members contribute to the success of the group and to address any potential issues early on
\begin{itemize}
\item If you feel that there are issues within your team, you are encouraged to discuss it with your team members and to bring it to my or your TA's attention ASAP (don't wait till things get worse)
\end{itemize}

\end{itemize}

\end{frame}

%%%%%%%%%%%%%%%%%%%%%%%%%%%%%%%%%%%

\begin{frame}
\frametitle{Attendance \& participation}

\red{Objective:} Make you an active participant and help me pace the class 

\begin{itemize}

\item Attendance and participation during class, as well as your activity on Piazza make up a non-insignificant portion of your grade in this class

\item Might sometimes call on you during the class discussion, however it is your responsibility to be an active participant without being called on

\item Up to two excused absences (regardless of reason)

\end{itemize}

\end{frame}

%%%%%%%%%%%%%%%%%%%%%%%%%%%%%%%%%%%

\begin{frame}
\frametitle{Problem sets (PS)}

\red{Objective:} Help you develop a more in-depth understanding of the material and help you prepare for exams and projects

\begin{itemize}

\item Due Monday nights

\item Questions from the textbook

\item Show \emph{all} your work to receive credit

\item Submit on Sakai
\begin{itemize}
\item PDF submission ensures we receive and see what you intended to submit
\item Word submission is less reliable
\item Writing your answers in the text box is acceptable but make sure to save often (Sakai doesn't automatically save for you)
\end{itemize}

\item Welcomed and encouraged to work with others, but turn in your own work

\item No make-ups, excused absences (e.g. STINF) do not excuse problem sets

\item Lowest PS score will be dropped

\end{itemize}

\end{frame}

%%%%%%%%%%%%%%%%%%%%%%%%%%%%%%%%%%%

\begin{frame}[fragile]
\frametitle{Labs}

\red{Objective:} Give you hands on experience with data analysis using statistical software and provide you with tools for the projects

\begin{itemize}

\item Software: R/RStudio accessed on the web via NetID

\item Due dates on course website

\item No make-ups, excused absences (e.g. STINF) do not excuse labs

\item Lowest lab score will be dropped

\end{itemize}

\end{frame}

%%%%%%%%%%%%%%%%%%%%%%%%%%%%%%%%%%%

\begin{frame}
\frametitle{Readiness assessments (RA)}

\red{Objective:} Encourage you to watch the videos and/or complete the reading assignment and review the learning objectives prior to coming to class as well as evaluate your conceptual understanding of the unit's material

\begin{itemize}

\item Dates on course website

\item 10 multiple choice questions, at the beginning of a unit

\item Conceptual questions addressing the learning objectives of the new unit, assessing familiarity and reasoning, not mastery

\item Take the individual RA on Sakai, then re-take in teams

\item Individual RA score 3/4 of grade, team RA score 1/4 \& your input during the team portion will factor into your participation grade

\item Lowest RA score will be dropped

\end{itemize}

\end{frame}

%%%%%%%%%%%%%%%%%%%%%%%%%%%%%%%%%%%

\begin{frame}
\frametitle{Performance assessments (PA)}

\red{Objective:} Evaluate your mastery of the material by the end of a unit and give you instant feedback on your performance.

\begin{itemize}

\item Due dates on course website

\item 10 multiple choice questions, at the end of a unit

\item Taken individually on Sakai

\item Lowest PA score will be dropped

\end{itemize}

\end{frame}

%%%%%%%%%%%%%%%%%%%%%%%%%%%%%%%%%%%

\begin{frame}
\frametitle{Projects}

\red{Objective:} Give you independent applied research experience using real data and statistical methods

\begin{itemize}

\item For a parameter of interest to you, you will describe the relevant data, compute a confidence interval and conduct a hypothesis test, and summarize your findings in a written, fully reproducible, data analysis report

\item Must complete both project and score at least 30\% of the points on each project in order to pass this class

\end{itemize}

\end{frame}

%%%%%%%%%%%%%%%%%%%%%%%%%%%%%%%%%%%

\begin{frame}
\frametitle{Exams}

\begin{center}
\rowcolors{1}{}{custom_lightGray}
\renewcommand\arraystretch{1.25}
{\footnotesize
\begin{tabular}{ r | l }
Exam								& Date \\
\hline
Midterm 								& \ExamDate \\    
Final 								& \FinalDate   
\end{tabular}
}
\end{center}

\begin{itemize}

\item Exam dates cannot be changed, no make-up exams will be given

\item If you cannot take the exams on these dates you should drop this class

\item If you have a medical excuse on the day of an exam, work with your Dean

\item Calculator + cheat sheet allowed

\end{itemize}

\end{frame}

%%%%%%%%%%%%%%%%%%%%%%%%%%%%%%%%%%%

\section{Support}

%%%%%%%%%%%%%%%%%%%%%%%%%%%%%%%%%%%

\begin{frame}
\frametitle{Email \& Piazza}

\begin{itemize}

\item I will regularly send announcements by email, so make sure to check your email  daily

\item Any \emph{non-personal} questions related to the material covered in class, problem sets, labs, projects, etc. should be posted on Piazza forum

\item Before posting a new question please make sure to check if your question has already been answered, and answer others' questions

\item Use informative titles for your posts

\item It is more efficient to answer most statistical questions ``in person" so make use of OH

\end{itemize}

\end{frame}

%%%%%%%%%%%%%%%%%%%%%%%%%%%%%%%%%%%

\begin{frame}
\frametitle{Office Hours}

\begin{itemize}
\item Prof. \LastName{}: \OfficeHours{}
\item Andrew: Generally MW 12:30 - 1:30pm
\item Changes will be announced at the beginning of the week
\item This week: No OH today, but I'll stick around to answer questions after class, Andrew will hold office hours tomorrow (Thursday) 12:30-1:30pm, and we will both answer questions on Piazza throughout the week
\end{itemize}

\end{frame}

%%%%%%%%%%%%%%%%%%%%%%%%%%%%%%%%%%%

\begin{frame}
\frametitle{Students with disabilities}

Students with disabilities who believe they may need accommodations in this class are encouraged to contact the \webLink{http://www.access.duke.edu/students/requesting/index.php}{Student Disability Access Office} at (919) 668-1267 as soon as possible to better ensure that such accommodations can be made

\vfill

\ct{\webURL{http://www.access.duke.edu/students/requesting/index.php}}

\end{frame}

%%%%%%%%%%%%%%%%%%%%%%%%%%%%%%%%%%%

\section{Policies}

%%%%%%%%%%%%%%%%%%%%%%%%%%%%%%%%%%%

\begin{frame}
\frametitle{Late work policy}

\begin{itemize}

\item Late work policy for problem sets and labs reports:
\begin{itemize}
\item next day: lose 30\% of points (within 24 hours of due date)
\item later than next day: lose all points
\end{itemize}

\item Late work policy for project: 20\% off for each day late

\end{itemize}

\end{frame}

%%%%%%%%%%%%%%%%%%%%%%%%%%%%%%%%%%%

\begin{frame}
\frametitle{Regrade policy}

Regrade requests must be made \hl{within 2 days} of when the assignment is returned, and must be submitted to me in writing 

\begin{itemize}

\item These will be honored if points were tallied incorrectly, or if you feel your answer is correct but it was marked wrong

\item No regrade will be made to alter the number of points deducted for a mistake

\item There will be no grade changes after the final exam

\end{itemize}

\end{frame}

%%%%%%%%%%%%%%%%%%%%%%%%%%%%%%%%%%%

\begin{frame}
\frametitle{Make up policy}

\begin{itemize}

\item No make-up for attendance, individual and team readiness assessments, labs, problem sets, projects, or exams

\item If the midterm exam must be missed due to a documented medical excuse, absence must be officially excused \hl{in advance}, in which case the missing exam score will be imputed using the final exam score

\item The final exam must be taken at the stated time

\item You must take the final exam and turn in the projects in order to pass this course

\end{itemize}

\end{frame}

%%%%%%%%%%%%%%%%%%%%%%%%%%%%%%%%%%%

\begin{frame}
\frametitle{Other policies}

\begin{itemize}

\item Use of disallowed materials (textbook, class notes, web references, any form of communication with classmates or other persons, etc.) during exams will not be tolerated

\end{itemize}

\end{frame}

%%%%%%%%%%%%%%%%%%%%%%%%%%%%%%%%%%%%

\begin{frame}
\frametitle{Academic Dishonesty}

Any form of academic dishonesty will result in an immediate 0 on the given assignment and will be reported to the Office of Student Conduct. Additional penalties may also be assessed if deemed appropriate. If you have any questions about whether something is or is not allowed, ask me beforehand.

Some examples:

\begin{itemize}

\item Use of disallowed materials (including any form of communication with classmates or accessing the web) during exams and readiness assessments

\item Plagiarism of any kind

\item Use of outside answer keys or solution manuals for the homework

\end{itemize}

\end{frame}

%%%%%%%%%%%%%%%%%%%%%%%%%%%%%%%%%%%%

\section{Tips for success}

%%%%%%%%%%%%%%%%%%%%%%%%%%%%%%%%%%%%

\begin{frame}
\frametitle{Tips for success}

{\footnotesize
\begin{itemize}[<alert@+>]
\item Complete the reading before a new unit begins, and then review again after the unit is over.
\item Be an active participant during class.
\item Ask questions - during class or office hours, or by email. Ask me, your TA, and your classmates.
\item Do the problem sets - start early and make sure you attempt and understand all questions.
\item Start your projects early and and allow adequate time to complete them.
\item Give yourself plenty of time time to prepare a good cheat sheet for exams. This requires going through the material and taking the time to review the concepts that you're not comfortable with.
\item Do not procrastinate - don't let a unit go by with unanswered questions as it will just make the following unit's material even more difficult to follow. 
\end{itemize}
}

\end{frame}

%%%%%%%%%%%%%%%%%%%%%%%%%%%%%%%%%%%%

\end{document}