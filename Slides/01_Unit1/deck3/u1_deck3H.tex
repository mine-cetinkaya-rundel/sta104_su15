% -*- TeX-engine: xetex; eval: (auto-fill-mode 0); eval: (visual-line-mode 1); -*-
% Compile with XeLaTeX

%%%%%%%%%%%%%%%%%%%%%%%
% To do before class
%%%%%%%%%%%%%%%%%%%%%%%

% Print off Readiness Assessment 1

% Send email about registering clicker.
% Test run readiness assessment on iClicker.
% I need to get scratch off sheet from Mine.

% Send the Logistics/Week0Annoucnement (the night before).
% Send an email reminding students to bring a charged computer (the night before).

% Questions for Mine
% Can I get scratch off sheets.
% What do you do during group portion?
% Question: voluntary vs. non-response

%%%%%%%%%%%%%%%%%%%%%%%
% Option 1: Slides: (comment for handouts)   %
%%%%%%%%%%%%%%%%%%%%%%%

%\documentclass[slidestop,compress,mathserif,12pt,t,professionalfonts,xcolor=table]{beamer}
%
%% solution stuff
%\newcommand{\solnMult}[1]{
%\only<1>{#1}
%\only<2->{\red{\textbf{#1}}}
%}
%\newcommand{\soln}[1]{\textit{#1}}

%%%%%%%%%%%%%%%%%%%%%%%%%%%%%%%
% Option 2: Handouts, without solutions (post before class)    %
%%%%%%%%%%%%%%%%%%%%%%%%%%%%%%%

 \documentclass[11pt,containsverbatim,handout,xcolor=xelatex,dvipsnames,table]{beamer}

 % handout layout
 \usepackage{pgfpages}
 \pgfpagesuselayout{4 on 1}[letterpaper,landscape,border shrink=5mm]

 % solution stuff
 \newcommand{\solnMult}[1]{#1}
 \newcommand{\soln}[1]{}

 % % This breaks things for me for some reason.
 % tell pgfpages how to set page sizes in XeLaTeX
\renewcommand\pgfsetupphysicalpagesizes{%
    \pdfpagewidth\pgfphysicalwidth\pdfpageheight\pgfphysicalheight%
}

%%%%%%%%%%%%%%%%%%%%%%%%%%%%%%%%%%%%
% Option 3: Handouts, with solutions (may post after class if need be)    %
%%%%%%%%%%%%%%%%%%%%%%%%%%%%%%%%%%%%

% \documentclass[11pt,containsverbatim,handout,xcolor=xelatex,dvipsnames,table]{beamer}

% % handout layout
% \usepackage{pgfpages}
% \pgfpagesuselayout{4 on 1}[letterpaper,landscape,border shrink=5mm]

% % solution stuff
% \newcommand{\solnMult}[1]{\red{\textbf{#1}}}
% \newcommand{\soln}[1]{\textit{#1}}

% % % This breaks things for me for some reason.
% % % tell pgfpages how to set page sizes in XeLaTeX
% % \renewcommand\pgfsetupphysicalpagesizes{%
% %    \pdfpagewidth\pgfphysicalwidth\pdfpageheight\pgfphysicalheight%
% % }

%%%%%%%%%%%%%%%%%%%%%%%%%%%%%%%
% Option 4: Notes Only
%%%%%%%%%%%%%%%%%%%%%%%%%%%%%%%

% % See http://tex.stackexchange.com/questions/114219/add-notes-to-latex-beamer
% \documentclass[10pt,containsverbatim,xcolor=xelatex,dvipsnames,table,notes=only]{beamer}

% % handout layout
% \usepackage{pgfpages}
% \pgfpagesuselayout{2 on 1}[letterpaper, landscape, border shrink=5mm]

% % solution stuff
% \newcommand{\solnMult}[1]{#1}
% \newcommand{\soln}[1]{}

% % % Having a problem with this.
% % tell pgfpages how to set page sizes in XeLaTeX
% % \renewcommand\pgfsetupphysicalpagesizes{%
% %   \pdfpagewidth\pgfphysicalwidth\pdfpageheight\pgfphysicalheight%
% %}

%%%%%%%%%%
% Load style file, defaults  %
%%%%%%%%%%

\input{../../lec_style.tex}
% You cannot use numbers when defining variables.  Hence the use of letters, A, B, C, etc.

% Personal Info
\newcommand{\FirstName}{Mine}
\newcommand{\LastName}{\c{C}etinkaya-Rundel}
\newcommand{\OfficeHours}{Generally TR 12:30 - 1:30pm}

% Electronic Info
\newcommand{\PersonalSite}{http://stat.duke.edu/~mc301}
\newcommand{\CourseSite}{http://bit.ly/sta104su15}
\newcommand{\Email}{mine@stat.duke.edu}

% TAs
\newcommand{\TAA}{Andrew Wong}

% Exam Dates
\newcommand{\ExamDate}{May 29, 11am - 12:30pm (in class)}
\newcommand{\FinalDate}{June 24, 11am - 2pm}

% ALT ALT
% \input{../../definitions_custom.tex}

%%%%%%%%%%%
% Cover slide info    %
%%%%%%%%%%%

\title{Unit 1: Introduction to data}
\subtitle{3. Introduction to statistical inference}
\author{Sta 104 - Summer 2015}
\date{May 18, 2015}
\institute{Duke University, Department of Statistical Science}


%%%%%%%%%%%%%%%%%%%%%%%%%
% Begin document and set Helvetica Neue font   %
%%%%%%%%%%%%%%%%%%%%%%%%%

\begin{document}
\fontspec[Ligatures=TeX]{Helvetica Neue Light}

%%%%%%%%%%%%%%%%%%%%%%%%%%%%%%%%%%%

% Title Page

\begin{frame}[plain]

\titlepage
\vfill
{\scriptsize \webLink{\PersonalSite}{Dr. \LastName{}} \hfill Slides posted at  \webLink{\CourseSite}{\CourseSite}}
\addtocounter{framenumber}{-1} 

\end{frame}

%%%%%%%%%%%%%%%%%%%%%%%%%%%%%%%%%%%

\section{Housekeeping}

%%%%%%%%%%%%%%%%%%%%%%%%%%%%%%%%%%%

\begin{frame}
\frametitle{Announcements}

\begin{itemize}

\item PA 1 due tonight

\end{itemize}

\end{frame}

%%%%%%%%%%%%%%%%%%%%%%%%%%%%%%%%%%%

\begin{frame}
\frametitle{Fram last time...}

\vfill

Finish up content from last time...

\vfill

\end{frame}

%%%%%%%%%%%%%%%%%%%%%%%%%%%%%%%%%%%

\section{Case study: is yawning contagious?}

%%%%%%%%%%%%%%%%%%%%%%%%%%%%%%%%%%%%

\begin{frame}

\begin{columns}

\column{0.5\textwidth}

\includegraphics[width=0.9\textwidth]{figures/yawn/yawn1}\\
\includegraphics[width=0.9\textwidth]{figures/yawn/yawn2}

\column{0.5\textwidth}

\clicker{Do you think yawning is contagious?}

\begin{enumerate}[(a)]
\item Yes
\item No
\item Don't know
\end{enumerate}

\end{columns}

\end{frame}

%%%%%%%%%%%%%%%%%%%%%%%%%%%%%%%%%%%

\begin{frame}
\frametitle{Is yawning contagious?}

An experiment conducted by the MythBusters tested if a person can be subconsciously influenced into yawning if another person near them yawns. \\

\begin{center}
\includegraphics[width=0.7\textwidth]{figures/mythbusters}\\
\end{center}

\ct{\webURL{http://www.discovery.com/tv-shows/mythbusters/videos/is-yawning-contagious-minimyth.htm}}

\end{frame}


%%%%%%%%%%%%%%%%%%%%%%%%%%%%%%%%%%%

\begin{frame}
\frametitle{Experiment summary}

50 people were randomly assigned to two groups: 
\begin{itemize}
\item treatment: see someone yawn, $n = 34$
\item control: don't see someone yawn, $n = 16$
\end{itemize}

\begin{center}
\renewcommand\arraystretch{1.5}
\begin{tabular}{ l | c | c | c }
			& Treatment			& Control			& Total \\
\hline
Yawn		& 10				& 4				& 14 \\
Not Yawn		& 24				& 12				& 36 \\
\hline
Total			& 34				& 16				& 50 \\
\hline
\% Yawners	& \soln{\onslide<2->{$\frac{10}{34} = 0.29$}}	& \soln{\onslide<3->{$\frac{4}{16} = 0.25$}}	& 
\end{tabular}
\end{center}

\onslide<4->{

\disc{Based on the proportions we calculated, do you think yawning is really contagious, i.e. are seeing someone yawn and yawning dependent?}

}

\end{frame}

%%%%%%%%%%%%%%%%%%%%%%%%%%%%%%%%%%%

\begin{frame}
\frametitle{Dependence, or another possible explanation?}

\begin{itemize}

\item The observed differences might suggest that yawning is contagious, i.e. seeing someone yawn and yawning are dependent

\pause

\item But the differences are small enough that we might wonder if they might simple be \hl{due to chance}

\pause

\item Perhaps if we were to repeat the experiment, we would see slightly different results

\pause

\item So we will do just that - well, somewhat - and see what happens

\pause

\item Instead of actually conducting the experiment many times, we will \hl{simulate} our results

\end{itemize}

\end{frame}

%%%%%%%%%%%%%%%%%%%%%%%%%%%%%%%%%%%%

\subsection{Competing claims}

%%%%%%%%%%%%%%%%%%%%%%%%%%%%%%%%%%%%%

\begin{frame}
\frametitle{Two competing claims}

\begin{enumerate}

\item ``There is nothing going on." \\
Seeing someone yawn and yawning are \hl{independent}, observed difference in proportions of yawners in the treatment and control is simply due to chance. $\rightarrow$ \hl{Null hypothesis}

\pause

\item ``There is something going on." \\
Seeing someone yawn and yawning are \hl{dependent}, observed difference in proportions of yawners in the treatment and control is not due to chance. $\rightarrow$ \hl{Alternative hypothesis}

\end{enumerate}

\end{frame}

%%%%%%%%%%%%%%%%%%%%%%%%%%%%%%%%%%%%

\begin{frame}
\frametitle{A trial as a hypothesis test}

\begin{center}
\includegraphics[width=\textwidth]{figures/trial}
\end{center}

\begin{itemize}

\item $H_0$: Defendant is innocent 

\item $H_A$: Defendant is guilty

\item Present the evidence: collect data.

\item Judge the evidence: ``Could these data plausibly have happened by chance if the null hypothesis were true?"

\item Make a decision: ``How unlikely is unlikely?"

\end{itemize}

\end{frame}

%%%%%%%%%%%%%%%%%%%%%%%%%%%%%%%%%%%%%

\begin{frame}
\frametitle{Recap: hypothesis testing framework}

\begin{itemize}
\item Start with a \hl{null hypothesis ($H_0$)} (the status quo)

\pause

\item Set an \hl{alternative hypothesis ($H_A$)} (the research question, i.e. what we're testing for)

\pause

\item Conduct a hypothesis test under the assumption that the null hypothesis is true, either via simulation (today) or theoretical methods (later in the course)

\pause

\item Test results suggest the observed data are 
\begin{itemize}
\item reasonably likely to have occurred even if the null is true $\rightarrow$ stick with the null hypothesis (fail to reject $H_0$)
\item unlikely to have occurred even if the null is true $\rightarrow$ reject $H_0$ in favor of $H_A$
\end{itemize}

\pause

\item Never declare the null hypothesis to be true, because we simply do not know whether it's true or not $\rightarrow$ never ``accept the null hypothesis"

\end{itemize}

\end{frame}

%%%%%%%%%%%%%%%%%%%%%%%%%%%%%%%%%%%%%

\subsection{Testing via simulation}

%%%%%%%%%%%%%%%%%%%%%%%%%%%%%%%%%%%%%

\begin{frame}
\frametitle{Simulating the experiment...}

... under the assumption of independence, i.e. leaving things up to chance: \\

\pause

\begin{itemize}

\item Results from the simulations based on the \hl{null hypothesis} look like the data $\rightarrow$ difference between the proportions of yawners in the treatment and control groups was simply \hl{due to chance} (yawning is not contagious) \\

\pause

\item Results from the simulations based on the chance model do not look like the data $\rightarrow$ difference between the proportions of yawners in the treatment and control groups was not due to chance, but \hl{due to an actual effect of seeding} (yawning is contagious)

\end{itemize}

\end{frame}

%%%%%%%%%%%%%%%%%%%%%%%%%%%%%%%%%%%%

\begin{frame}
\frametitle{Simulation setup}

\begin{itemize}

\item A regular deck of cards is comprised of 52 cards: 4 aces, 4 of numbers 2-10, 4 jacks, 4 queens, and 4 kings.

\item Take out two aces from the deck of cards and set them aside.

\item The remaining 50 playing cards to represent each participant in the study:
\begin{itemize}
\item 14 face cards (including the 2 aces) represent the people who yawn.
\item 36 non-face cards represent the people who don't yawn.
\end{itemize}

\end{itemize}

\end{frame}

%%%%%%%%%%%%%%%%%%%%%%%%%%%%%%%%%%%

\begin{frame}
\frametitle{Running the simulation}

\begin{enumerate}

\item Shuffle the 50 cards at least 7 times to ensure that the cards counted out are from a random process

\item Divide the cards into two decks:
\begin{itemize}
\item deck 1: 16 cards $\rightarrow$ control
\item deck 2: 34 cards $\rightarrow$ treatment
\end{itemize}

\item Count the number of face cards (yawners) in each deck

\item Calculate the difference in proportions of yawners \red{(treatment - control)}, and submit this value using your clicker - \textbf{only one submission per team per simulation}

\item Repeat steps (1) - (4) many times

\end{enumerate}

\vfill

\ct{Why shuffle 7 times: \webURL{http://www.dartmouth.edu/~chance/course/topics/winning_number.html}}

\end{frame}

%%%%%%%%%%%%%%%%%%%%%%%%%%%%%%%%%%%

\subsection{Checking for independence}

\begin{frame}

\clicker{Do the simulation results suggest that yawning is contagious, i.e. does seeing someone yawn and yawning appear to be dependent? \\
(Hint: In the actual data the difference was 0.04, does this appear to be an unusual observation for the chance model?)}

\begin{multicols}{2}
\begin{enumerate}[(a)]
\item Yes
\item No
\end{enumerate}
\end{multicols}

\end{frame}

%%%%%%%%%%%%%%%%%%%%%%%%%%%%%%%%%%%%

\section{Case study: Tapping on caffeine}

%%%%%%%%%%%%%%%%%%%%%%%%%%%%%%%%%%%%

\begin{frame}
\frametitle{Tapping on caffeine}

\begin{itemize}

\item In a double-blind experiment a sample of male college students were asked to tap their fingers at a rapid rate. 

\item The sample was then divided at random into two groups of 10 students each. 

\item Each student drank the equivalent of about two cups of coffee, which included about 200 mg of caffeine for the students in one group but was decaffeinated coffee for the second group. 

\item After a two hour period, each student was tested to measure finger tapping rate (taps per minute). 

\end{itemize}

\end{frame}

%%%%%%%%%%%%%%%%%%%%%%%%%%%%%%%%%%%%

\begin{frame}
\frametitle{Data}

\begin{center}
\begin{tabular}{rrl}
  \hline
 & Taps & Group \\ 
  \hline
1 & 246 & Caffeine \\ 
  2 & 248 & Caffeine \\ 
  3 & 250 & Caffeine \\ 
  4 & 252 & Caffeine \\ 
  5 & 248 & Caffeine \\ 
  6 & 250 & Caffeine \\ 
$\cdots$ && \\
  16 & 248 & NoCaffeine \\ 
  17 & 242 & NoCaffeine \\ 
  18 & 244 & NoCaffeine \\ 
  19 & 246 & NoCaffeine \\ 
  20 & 242 & NoCaffeine \\ 
   \hline
\end{tabular}
\end{center}

\end{frame}

%%%%%%%%%%%%%%%%%%%%%%%%%%%%%%%%%%%%

\begin{frame}
\frametitle{}

\clicker{What type of plot would be useful to visualize the distributions of tapping rate in the caffeine and no caffeine groups.}

\begin{enumerate}[(a)]
\item Bar plot
\item Mosaic plot
\item Pie chart
\item \solnMult{Side-by-side box plots}
\item Single box plot
\end{enumerate}

\end{frame}

%%%%%%%%%%%%%%%%%%%%%%%%%%%%%%%%%%%%

\begin{frame}
\frametitle{Exploratory data analysis}

\twocol{0.4}{0.6}{
\disc{Compare the distributions of tapping rates in the caffeine and no caffeine groups.}
}
{
\begin{center}
{\small
\begin{tabular}{l | c | c | c}
\hline
		& Caffeine	& No Caffeine	& Difference \\
\hline
mean	&  248.3		& 244.8 		& 3.5 \\
SD		& 2.21		& 2.39 		& -0.18 \\
median	& 248		& 245 		& 3 \\
IQR		& 3.5			& 4.25		& -0.75 \\
\hline
\end{tabular}
}
\end{center}
}

\vspace{-1cm}

\begin{center}
\includegraphics[width=0.85\textwidth]{figures/caffeine/caffeinetaps_box.pdf}
\end{center}

\end{frame}

%%%%%%%%%%%%%%%%%%%%%%%%%%%%%%%%%%%%

\begin{frame}
\frametitle{}

\clicker{We are interested in finding out if caffeine increases tapping rate. Which of the following are the correct set of hypotheses?}

\begin{enumerate}[(a)]

\item $H_0: \mu_{caff} = \mu_{no~caff}$ \\
$H_A: \mu_{caff} < \mu_{no~caff}$

\item \solnMult{$H_0: \mu_{caff} = \mu_{no~caff}$ \\
$H_A: \mu_{caff} > \mu_{no~caff}$}

\item $H_0: \bar{x}_{caff} = \bar{x}_{no~caff}$ \\
$H_A: \bar{x}_{caff} > \bar{x}_{no~caff}$

\item $H_0: \mu_{caff} > \mu_{no~caff}$ \\
$H_A: \mu_{caff} = \mu_{no~caff}$

\item $H_0: \mu_{caff} = \mu_{no~caff}$ \\
$H_A: \mu_{caff} \ne \mu_{no~caff}$

\end{enumerate}

\end{frame}

%%%%%%%%%%%%%%%%%%%%%%%%%%%%%%%%%%%%

\begin{frame}
\frametitle{Simulation scheme}

\begin{itemize}

\item On 20 index cards write the tapping rate of each subject in the study.

\item Shuffle the cards and divide them into two stacks of 10 cards each, label one stack ``caffeine" and the other stack ``no caffeine".

\item Calculate the average tapping rates in the two simulated groups, and record the difference on a dot plot.

\item Repeat steps (2) and (3) many times to build a \hl{randomization distribution}.

\end{itemize}

\end{frame}

%%%%%%%%%%%%%%%%%%%%%%%%%%%%%%%%%%%%

\begin{frame}
\frametitle{Making a decision}

\vspace{-0.5cm}

\disc{Below is a randomization distribution of 100 simulated differences in means ($\bar{x}_c - \bar{x}_{nc}$). Calculate the p-value for the hypothesis test evaluating whether caffeine \underline{increases} average tapping rate.
\begin{center}
{\footnotesize
\begin{tabular}{l | c | c | c}
\hline
		& Caffeine	& No Caffeine	& Difference \\
\hline
mean	&  248.3		& 244.8 		& 3.5 \\
\hline
\end{tabular}
}
\end{center}
}

\begin{center}
\includegraphics[width=0.9\textwidth]{figures/caffeine/caffeinetaps_mean_ht.pdf}
\end{center}

\pause

\soln{1/100 = 0.01}

\end{frame}

%%%%%%%%%%%%%%%%%%%%%%%%%%%%%%%%%%%%

\begin{frame}
\frametitle{Testing for the median}

\disc{Describe how could we use the same approach to test whether the \underline{median} tapping rate is higher for the caffeine group?}

\soln{\only<2>{Use the same simulation scheme but record the difference between the medians instead of the means, and calculate the p-value as the proportion of simulations where the simulated difference in medians is at least 3.
}}

\end{frame}

%%%%%%%%%%%%%%%%%%%%%%%%%%%%%%%%%%%%

\begin{frame}
\frametitle{Testing for the median (cont.)}

\vspace{-0.5cm}

\disc{Below is a randomization distribution of 100 simulated differences in medians ($med_c - med_{nc}$). Do the data provide convincing evidence that caffeine increases \underline{median} tapping rate?
\begin{center}
{\footnotesize
\begin{tabular}{l | c | c | c}
\hline
		& Caffeine	& No Caffeine	& Difference \\
\hline
median	& 248		& 245 		& 3 \\
\hline
\end{tabular}
}
\end{center}
}

\begin{center}
\includegraphics[width=0.65\textwidth]{figures/caffeine/caffeinetaps_median_ht.pdf}
\end{center}

\end{frame}

%%%%%%%%%%%%%%%%%%%%%%%%%%%%%%%%%%%%

\begin{frame}
\frametitle{[time permitting]}

\vfill

\app{1.4 Randomization testing}{$\:$\\ See the course website for instructions. \\$\:$}

\vfill

\end{frame}

%%%%%%%%%%%%%%%%%%%%%%%%%%%%%%%%%%%%

\end{document}