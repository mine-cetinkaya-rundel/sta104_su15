% -*- TeX-engine: xetex; eval: (auto-fill-mode 0); eval: (visual-line-mode 1); -*-
% Compile with XeLaTeX

%%%%%%%%%%%%%%%%%%%%%%%
% To do before class
%%%%%%%%%%%%%%%%%%%%%%%

% Print off Readiness Assessment 1

% Send email about registering clicker.
% Test run readiness assessment on iClicker.
% I need to get scratch off sheet from Mine.

% Send the Logistics/Week0Annoucnement (the night before).
% Send an email reminding students to bring a charged computer (the night before).

% Questions for Mine
% Can I get scratch off sheets.
% What do you do during group portion?
% Question: voluntary vs. non-response

%%%%%%%%%%%%%%%%%%%%%%%
% Option 1: Slides: (comment for handouts)   %
%%%%%%%%%%%%%%%%%%%%%%%
%
%\documentclass[slidestop,compress,mathserif,12pt,t,professionalfonts,xcolor=table]{beamer}
%
%% solution stuff
%\newcommand{\solnMult}[1]{
%\only<1>{#1}
%\only<2->{\red{\textbf{#1}}}
%}
%\newcommand{\soln}[1]{\textit{#1}}

%%%%%%%%%%%%%%%%%%%%%%%%%%%%%%%
% Option 2: Handouts, without solutions (post before class)    %
%%%%%%%%%%%%%%%%%%%%%%%%%%%%%%%

 \documentclass[11pt,containsverbatim,handout,xcolor=xelatex,dvipsnames,table]{beamer}

 % handout layout
 \usepackage{pgfpages}
 \pgfpagesuselayout{4 on 1}[letterpaper,landscape,border shrink=5mm]

 % solution stuff
 \newcommand{\solnMult}[1]{#1}
 \newcommand{\soln}[1]{}

 % % This breaks things for me for some reason.
 % tell pgfpages how to set page sizes in XeLaTeX
 \renewcommand\pgfsetupphysicalpagesizes{%
    \pdfpagewidth\pgfphysicalwidth\pdfpageheight\pgfphysicalheight%
 }

%%%%%%%%%%%%%%%%%%%%%%%%%%%%%%%%%%%%
% Option 3: Handouts, with solutions (may post after class if need be)    %
%%%%%%%%%%%%%%%%%%%%%%%%%%%%%%%%%%%%

% \documentclass[11pt,containsverbatim,handout,xcolor=xelatex,dvipsnames,table]{beamer}

% % handout layout
% \usepackage{pgfpages}
% \pgfpagesuselayout{4 on 1}[letterpaper,landscape,border shrink=5mm]

% % solution stuff
% \newcommand{\solnMult}[1]{\red{\textbf{#1}}}
% \newcommand{\soln}[1]{\textit{#1}}

% % % This breaks things for me for some reason.
% % % tell pgfpages how to set page sizes in XeLaTeX
% % \renewcommand\pgfsetupphysicalpagesizes{%
% %    \pdfpagewidth\pgfphysicalwidth\pdfpageheight\pgfphysicalheight%
% % }

%%%%%%%%%%%%%%%%%%%%%%%%%%%%%%%
% Option 4: Notes Only
%%%%%%%%%%%%%%%%%%%%%%%%%%%%%%%

% % See http://tex.stackexchange.com/questions/114219/add-notes-to-latex-beamer
% \documentclass[10pt,containsverbatim,xcolor=xelatex,dvipsnames,table,notes=only]{beamer}

% % handout layout
% \usepackage{pgfpages}
% \pgfpagesuselayout{2 on 1}[letterpaper, landscape, border shrink=5mm]

% % solution stuff
% \newcommand{\solnMult}[1]{#1}
% \newcommand{\soln}[1]{}

% % % Having a problem with this.
% % tell pgfpages how to set page sizes in XeLaTeX
% % \renewcommand\pgfsetupphysicalpagesizes{%
% %   \pdfpagewidth\pgfphysicalwidth\pdfpageheight\pgfphysicalheight%
% %}

%%%%%%%%%%
% Load style file, defaults  %
%%%%%%%%%%

\input{../../lec_style.tex}
% You cannot use numbers when defining variables.  Hence the use of letters, A, B, C, etc.

% Personal Info
\newcommand{\FirstName}{Mine}
\newcommand{\LastName}{\c{C}etinkaya-Rundel}
\newcommand{\OfficeHours}{Generally TR 12:30 - 1:30pm}

% Electronic Info
\newcommand{\PersonalSite}{http://stat.duke.edu/~mc301}
\newcommand{\CourseSite}{http://bit.ly/sta104su15}
\newcommand{\Email}{mine@stat.duke.edu}

% TAs
\newcommand{\TAA}{Andrew Wong}

% Exam Dates
\newcommand{\ExamDate}{May 29, 11am - 12:30pm (in class)}
\newcommand{\FinalDate}{June 24, 11am - 2pm}

% ALT ALT
% \input{../../definitions_custom.tex}

%%%%%%%%%%%
% Cover slide info    %
%%%%%%%%%%%

\title{Unit 2: Probability and distributions}
\subtitle{1. Probability and conditional probability}
\author{Sta 104 - Summer 2015}
\date{May 19, 2015}
\institute{Duke University, Department of Statistical Science}


%%%%%%%%%%%%%%%%%%%%%%%%%
% Begin document and set Helvetica Neue font   %
%%%%%%%%%%%%%%%%%%%%%%%%%

\begin{document}
\fontspec[Ligatures=TeX]{Helvetica Neue Light}

%%%%%%%%%%%%%%%%%%%%%%%%%%%%%%%%%%%

% Title Page

\begin{frame}[plain]

\titlepage
\vfill
{\scriptsize \webLink{\PersonalSite}{Dr. \LastName{}} \hfill Slides posted at  \webLink{\CourseSite}{\CourseSite}}
\addtocounter{framenumber}{-1} 

\end{frame}

%%%%%%%%%%%%%%%%%%%%%%%%%%%%%%%%%%%

\section{Housekeeping}

%%%%%%%%%%%%%%%%%%%%%%%%%%%%%%%%%%%

\begin{frame}
\frametitle{Feedback from Lab and PS}

\begin{itemize}

\item Lab: Put your code in R chunks so that the markdown can process it as code and produce the desired output and plots.

\item PS1: 
\begin{itemize}
\item 1.6 (c): How is income recorded? (Under 2,600; 10,400 to 15,600; above 36,400; ...)
\item 1.14 (b): What type of a sample is it if you only ask your friends to respond?
\item 1.46 (c): Is the histogram or the intensity map more informative?
\end{itemize}

\end{itemize}

\end{frame}

%%%%%%%%%%%%%%%%%%%%%%%%%%%%%%%%%%%%

\begin{frame}
\frametitle{RA 2}

\begin{itemize}

\item 15 min individual

\item 10 min teams

\end{itemize}

\end{frame}

%%%%%%%%%%%%%%%%%%%%%%%%%%%%%%%%%%%

\section{Main ideas}

%%%%%%%%%%%%%%%%%%%%%%%%%%%%%%%%%%%%

\subsection{Disjoint and independent do not mean the same thing}
\label{mi1}

%%%%%%%%%%%%%%%%%%%%%%%%%%%%%%%%%%%%

\begin{frame}
\frametitle{1. Disjoint and independent do not mean the same thing}

\begin{itemize}

\item \hl{Disjoint (mutually exclusive) events} cannot happen at the same time
\begin{itemize}
\item A voter cannot register as a Democrat and a Republican at the same time
\item But s/he might be a Republican and a Moderate at the same time -- \hl{non-disjoint events}
\item For disjoint A and B: $P(A~and~B) = 0$
\end{itemize}

\pause

\item If A and B are \hl{independent events}, having information on A does not tell us anything about B (and vice versa)
\begin{itemize}
\item If A and B are independent: 
\begin{itemize}
\item $P(A~|~B) = P(A)$
\item $P(A~and~B) = P(A) \times P(B)$
\end{itemize}
\end{itemize}

\end{itemize}

\end{frame}

%%%%%%%%%%%%%%%%%%%%%%%%%%%%%%%%%%%%

\subsection{Application of the addition rule depends on disjointness of events}
\label{mi2}

%%%%%%%%%%%%%%%%%%%%%%%%%%%%%%%%%%%%

\begin{frame}
\frametitle{2. Application of the addition rule depends on disjointness of events}

\begin{itemize}

\item \hl{General addition rule:} P(A or B) = P(A) + P(B) - P(A and B)

\item A or B = either A or B or both

\end{itemize}

\vspace{0.5cm}

\pause

\twocol{0.5}{0.5}{
\textbf{disjoint events:} \\
P(A or B) \\
= P(A) + P(B) - P(A and B) \\
= 0.4 + 0.3 - 0 = 0.7
\begin{center}
\includegraphics[width = 0.9\textwidth]{figures/venn/venn_disjoint}
\end{center}
}
{
\pause
\textbf{non-disjoint events:} \\
P(A or B) \\
= P(A) + P(B) - P(A and B) \\
= 0.4 + 0.3 - 0.02 = 0.68
\begin{center}
\includegraphics[width = 0.9\textwidth]{figures/venn/venn_non_disjoint}
\end{center}
}

\end{frame}

%%%%%%%%%%%%%%%%%%%%%%%%%%%%%%%%%%%%

\subsection{Bayes' theorem works for all types of events}
\label{mi3}

%%%%%%%%%%%%%%%%%%%%%%%%%%%%%%%%%%%%

\begin{frame}
\frametitle{3. Bayes' theorem works for all types of events}

\begin{itemize}

\item \hl{Bayes' theorem:} $P(A~|~B) = \frac{P(A~and~B)}{P(B)}$

\pause

\item ... can be rewritten as: $P(A~and~B) = P(A~|~B) \times P(B)$

\end{itemize}

\pause

\vspace{0.5cm}

\twocol{0.5}{0.5}{
\textbf{disjoint events:}

\begin{itemize}
\item We know P(A $|$ B) = 0, since if B happened A could not have happened
\pause
\item P(A and B) \\
= P(A $|$ B) $\times$ P(B) \\
\pause
= \red{0} $\times$ P(B) = 0
\end{itemize}
}
{
\pause

\textbf{independent events:}

\begin{itemize}
\item We know P(A $|$ B) = P(A), since knowing B doesn't tell us anything about A
\pause
\item P(A and B) \\
= P(A $|$ B) $\times$ P(B) \\
\pause
= \red{P(A)} $\times$ P(B)
\end{itemize}
}

\end{frame}

%%%%%%%%%%%%%%%%%%%%%%%%%%%%%%%%%%%%

\begin{frame}
\frametitle{}

\vfill

\app{2.1 Probability and conditional probability}{$\:$\\ See the course website for instructions. \\$\:$}

\vfill

\end{frame}

%%%%%%%%%%%%%%%%%%%%%%%%%%%%%%%%%%%%

\section{Summary}

%%%%%%%%%%%%%%%%%%%%%%%%%%%%%%%%%%%%

\begin{frame}
\frametitle{Summary of main ideas}

\vfill

\begin{enumerate}

\item \nameref{mi1}

\item \nameref{mi2}

\item \nameref{mi3}

\end{enumerate}

\vfill

\end{frame}

%%%%%%%%%%%%%%%%%%%%%%%%%%%%%%%%%%%

\end{document}