% -*- TeX-engine: xetex; eval: (auto-fill-mode 0); eval: (visual-line-mode 1); -*-
% Compile with XeLaTeX

%%%%%%%%%%%%%%%%%%%%%%%
% To do before class
%%%%%%%%%%%%%%%%%%%%%%%

% Print off Readiness Assessment 1

% Send email about registering clicker.
% Test run readiness assessment on iClicker.
% I need to get scratch off sheet from Mine.

% Send the Logistics/Week0Annoucnement (the night before).
% Send an email reminding students to bring a charged computer (the night before).

% Questions for Mine
% Can I get scratch off sheets.
% What do you do during group portion?
% Question: voluntary vs. non-response

%%%%%%%%%%%%%%%%%%%%%%%
% Option 1: Slides: (comment for handouts)   %
%%%%%%%%%%%%%%%%%%%%%%%

%\documentclass[slidestop,compress,mathserif,12pt,t,professionalfonts,xcolor=table]{beamer}
%
%% solution stuff
%\newcommand{\solnMult}[1]{
%\only<1>{#1}
%\only<2->{\red{\textbf{#1}}}
%}
%\newcommand{\soln}[1]{\textit{#1}}

%%%%%%%%%%%%%%%%%%%%%%%%%%%%%%%
% Option 2: Handouts, without solutions (post before class)    %
%%%%%%%%%%%%%%%%%%%%%%%%%%%%%%%

 \documentclass[11pt,containsverbatim,handout,xcolor=xelatex,dvipsnames,table]{beamer}

 % handout layout
 \usepackage{pgfpages}
 \pgfpagesuselayout{4 on 1}[letterpaper,landscape,border shrink=5mm]

 % solution stuff
 \newcommand{\solnMult}[1]{#1}
 \newcommand{\soln}[1]{}

 % % This breaks things for me for some reason.
 % tell pgfpages how to set page sizes in XeLaTeX
% \renewcommand\pgfsetupphysicalpagesizes{%
%    \pdfpagewidth\pgfphysicalwidth\pdfpageheight\pgfphysicalheight%
% }

%%%%%%%%%%%%%%%%%%%%%%%%%%%%%%%%%%%%
% Option 3: Handouts, with solutions (may post after class if need be)    %
%%%%%%%%%%%%%%%%%%%%%%%%%%%%%%%%%%%%

% \documentclass[11pt,containsverbatim,handout,xcolor=xelatex,dvipsnames,table]{beamer}

% % handout layout
% \usepackage{pgfpages}
% \pgfpagesuselayout{4 on 1}[letterpaper,landscape,border shrink=5mm]

% % solution stuff
% \newcommand{\solnMult}[1]{\red{\textbf{#1}}}
% \newcommand{\soln}[1]{\textit{#1}}

% % % This breaks things for me for some reason.
% % % tell pgfpages how to set page sizes in XeLaTeX
% % \renewcommand\pgfsetupphysicalpagesizes{%
% %    \pdfpagewidth\pgfphysicalwidth\pdfpageheight\pgfphysicalheight%
% % }

%%%%%%%%%%%%%%%%%%%%%%%%%%%%%%%
% Option 4: Notes Only
%%%%%%%%%%%%%%%%%%%%%%%%%%%%%%%

% % See http://tex.stackexchange.com/questions/114219/add-notes-to-latex-beamer
% \documentclass[10pt,containsverbatim,xcolor=xelatex,dvipsnames,table,notes=only]{beamer}

% % handout layout
% \usepackage{pgfpages}
% \pgfpagesuselayout{2 on 1}[letterpaper, landscape, border shrink=5mm]

% % solution stuff
% \newcommand{\solnMult}[1]{#1}
% \newcommand{\soln}[1]{}

% % % Having a problem with this.
% % tell pgfpages how to set page sizes in XeLaTeX
% % \renewcommand\pgfsetupphysicalpagesizes{%
% %   \pdfpagewidth\pgfphysicalwidth\pdfpageheight\pgfphysicalheight%
% %}

%%%%%%%%%%
% Load style file, defaults  %
%%%%%%%%%%

\input{../../lec_style.tex}
% You cannot use numbers when defining variables.  Hence the use of letters, A, B, C, etc.

% Personal Info
\newcommand{\FirstName}{Mine}
\newcommand{\LastName}{\c{C}etinkaya-Rundel}
\newcommand{\OfficeHours}{Generally TR 12:30 - 1:30pm}

% Electronic Info
\newcommand{\PersonalSite}{http://stat.duke.edu/~mc301}
\newcommand{\CourseSite}{http://bit.ly/sta104su15}
\newcommand{\Email}{mine@stat.duke.edu}

% TAs
\newcommand{\TAA}{Andrew Wong}

% Exam Dates
\newcommand{\ExamDate}{May 29, 11am - 12:30pm (in class)}
\newcommand{\FinalDate}{June 24, 11am - 2pm}

% ALT ALT
% \input{../../definitions_custom.tex}

%%%%%%%%%%%
% Cover slide info    %
%%%%%%%%%%%

\title{Unit 3: Foundations for inference}
\subtitle{1. Variability in estimates and CLT}
\author{Sta 104 - Summer 2015}
\date{May 26, 2015}
\institute{Duke University, Department of Statistical Science}


%%%%%%%%%%%%%%%%%%%%%%%%%
% Begin document and set Helvetica Neue font   %
%%%%%%%%%%%%%%%%%%%%%%%%%

\begin{document}
\fontspec[Ligatures=TeX]{Helvetica Neue Light}

%%%%%%%%%%%%%%%%%%%%%%%%%%%%%%%%%%%

% Title Page

\begin{frame}[plain]

\titlepage
\vfill
{\scriptsize \webLink{\PersonalSite}{Dr. \LastName{}} \hfill Slides posted at  \webLink{\CourseSite}{\CourseSite}}
\addtocounter{framenumber}{-1} 

\end{frame}

%%%%%%%%%%%%%%%%%%%%%%%%%%%%%%%%%%%

\section{Housekeeping}

%%%%%%%%%%%%%%%%%%%%%%%%%%%%%%%%%%%

\begin{frame}
\frametitle{Announcements}

\begin{itemize}

\item Review session 1-2pm today

\item Review materials posted on course website + review quizzes on Coursera (not graded)
\begin{center}
\includegraphics[width=0.8\textwidth]{figures/coursera_quiz}
\end{center}

\end{itemize}

\end{frame}

%%%%%%%%%%%%%%%%%%%%%%%%%%%%%%%%%%%

\begin{frame}
\frametitle{RA 3}

\begin{itemize}

\item Individual - 15 mins

\item Team - 10 mins

\end{itemize}

\end{frame}

%%%%%%%%%%%%%%%%%%%%%%%%%%%%%%%%%%%

\section{Main ideas}

%%%%%%%%%%%%%%%%%%%%%%%%%%%%%%%%%%%%

\subsection{Sample statistics vary from sample to sample}
\label{mi1}

%%%%%%%%%%%%%%%%%%%%%%%%%%%%%%%%%%%%

\begin{frame}
\frametitle{Sample statistics vary from sample to sample}

\begin{itemize}

\item We are often interested in \hl{population parameters}.

\item Since complete populations are difficult (or impossible) to collect data on, we use \hl{sample statistics} as \hl{point estimates} for the unknown population parameters of interest.

\item Sample statistics vary from sample to sample.

\item Quantifying how sample statistics vary provides a way to estimate the \hl{margin of error} associated with our point estimate.

\item But before we get to quantifying the variability among samples, let's try to understand how and why point estimates vary from sample to sample.

\end{itemize}

% ALT ALT
% \pause

\disc{Suppose we randomly sample 1,000 adults from each state in the US. Would you expect the sample means of their ages to be the same, somewhat different, or very different?}

%---Note---%
\note{

Pop parameter: what is the average GPA at Duke?

}

\end{frame}

%%%%%%%%%%%%%%%%%%%%%%%%%%%%%%%%%%%

\begin{frame}
\frametitle{}

\disc{We would like to estimate the average number of drinks it takes students to get drunk. 
\begin{itemize}
\item We will assume that our population is comprised of 146 students.
\item Assume also that we don't have the resources to collect data from all 146, so we will take a sample of size $n = 10$. 
\end{itemize}
If we randomly select observations from this data set, which values are most likely to be selected, which are least likely?}

\begin{center}
\includegraphics[width=0.6\textwidth]{figures/no_drinks_drunk/hist_no_drinks_drunk_pop} 
\end{center}

%---Note---%
\note{

}

\end{frame}

%%%%%%%%%%%%%%%%%%%%%%%%%%%%%%%%%%%

\begin{frame}[fragile]
\frametitle{}

\begin{itemize}

\item Sample, with replacement, ten student IDs:
{\footnotesize
\begin{Verbatim}[frame=single, formatcom=\color{blue}]
> sample(1:146, size = 10, replace = TRUE)
\end{Verbatim}
}
\pause
{\footnotesize
\begin{Verbatim}[frame=single, formatcom=\color{gray}]
[1]  59 121  88  46  58  72  82  81   5  10
\end{Verbatim}
}
\pause

\item Find the students with these IDs:

\begin{center}
\includegraphics[width=0.65\textwidth]{figures/no_drinks_drunk/no_drinks_drunk_mysample}
\end{center}

\pause

\item Calculate the sample mean: $(8+6+10+4+5+3+5+6+6+6) / 10 = 5.9$

\end{itemize}

%---Note---%
\note{

}

\end{frame}

%%%%%%%%%%%%%%%%%%%%%%%%%%%%%%%%%%%%

\begin{frame}[fragile]
\frametitle{}

\activity{Creating a sampling distribution}{Repeat this, and report your sample mean.}
\begin{enumerate}

\item Sample, with replacement, ten student IDs:
{\footnotesize
\begin{Verbatim}[frame=single, formatcom=\color{blue}]
> sample(1:146, size = 10, replace = TRUE)
\end{Verbatim}
}

\item Find the students with these IDs:

\begin{center}
\includegraphics[width=0.6\textwidth]{figures/no_drinks_drunk/no_drinks_drunk_clean}
\end{center}

\item Calculate the sample mean, round it to 2 decimal places, and report to me.

\end{enumerate}

\end{frame}

%%%%%%%%%%%%%%%%%%%%%%%%%%%%%%%%%%%

\begin{frame}[fragile]
\frametitle{Sampling distribution}

What you just constructed is called a \hl{sampling distribution}.

$\:$ \\
\disc{What is the shape and center of this distribution. Based on this distribution what do you think is the true population average?}

$\:$ \\
\soln{\only<2>{
5.39
}}

\end{frame}


%%%%%%%%%%%%%%%%%%%%%%%%%%%%%%%%%%%%

\begin{frame}[fragile]
\frametitle{Average number of Duke games attended}

Next let's look at the population data for the number of Duke basketball games attended:

\begin{center}
\includegraphics[width=0.8\textwidth]{figures/duke_games/hist_duke_games_pop}
\end{center}



\end{frame}

%%%%%%%%%%%%%%%%%%%%%%%%%%%%%%%%%%%%

\begin{frame}[fragile]
\frametitle{Average number of Duke games attended (cont.)}

Sampling distribution, n = 10:

\twocol{0.6}{0.4}{
\begin{center}
\includegraphics[width=\textwidth]{figures/duke_games/hist_duke_games_sampling10}
\end{center}
}
{
\disc{What does each observation in this distribution represent?}
\soln{\only<2->{Sample mean, $\bar{x}$, of samples of size $n = 10$.}}
\disc{Is the variability of the sampling distribution smaller or larger than the variability of the population distribution?}
\soln{\only<3->{Smaller, sample means will vary less than individual observations.}}
}

\end{frame}


%%%%%%%%%%%%%%%%%%%%%%%%%%%%%%%%%%%%

\begin{frame}[fragile]
\frametitle{Average number of Duke games attended (cont.)}

Sampling distribution, n = 30:

\twocol{0.6}{0.4}{
\begin{center}
\includegraphics[width=\textwidth]{figures/duke_games/hist_duke_games_sampling30}
\end{center}
}
{
\disc{How did the shape, center, and spread of the sampling distribution change going from $n = 10$ to $n = 30$?}
\soln{\only<2->{Shape is more symmetric, center is about the same, spread is smaller.}}
}



\end{frame}


%%%%%%%%%%%%%%%%%%%%%%%%%%%%%%%%%%%%

\begin{frame}[fragile]
\frametitle{Average number of Duke games attended (cont.)}

Sampling distribution, n = 70:

\begin{center}
\includegraphics[width=0.6\textwidth]{figures/duke_games/hist_duke_games_sampling70}
\end{center}

\end{frame}


%%%%%%%%%%%%%%%%%%%%%%%%%%%%%%%%%%%%

\begin{frame}[fragile]
\frametitle{Average number of Duke games attended (cont.)}

\clicker{The mean of the sampling distribution is 5.75, and the standard deviation of the sampling distribution (also called the \hl{standard error}) is 0.75. Which of the following is the most reasonable guess for the 95\% confidence interval for the true average number of Duke games attended by students?}

\begin{enumerate}[(a)]
\item $5.75 \pm 0.75$
\item \solnMult{$5.75 \pm 2 \times 0.75$} \soln{\only<2>{\red{$\rightarrow (4.25,7.25)$}}}
\item $5.75 \pm 3 \times 0.75$
\item cannot tell from the information given
\end{enumerate}

\end{frame}

%%%%%%%%%%%%%%%%%%%%%%%%%%%%%%%%%%%%

\subsection{CLT describes the shape, center, and spread of sampling distributions}
\label{mi2}

%%%%%%%%%%%%%%%%%%%%%%%%%%%%%%%%%%%%

\begin{frame}
\frametitle{2. CLT describes the shape, center, and spread of sampling distributions}

\emph{Under the right conditions}, the distribution of the sample means is well
approximated by a normal distribution:
\[ \bar{x} \sim N \pr{ mean = \mu, SE = \frac{\sigma}{\sqrt{n}} } \]
If $\sigma$ is unknown, use $s$.

\pause

\begin{itemize}

\item So it wasn't a coincidence that the sampling distributions we saw earlier were symmetric.

\item We won't go into the proving why $SE =  \frac{\sigma}{\sqrt{n}}$, but note that as $n$ increases $SE$ decreases. 

\item As the sample size increases we would expect samples to yield more consistent sample means, hence the variability among the sample means would be lower.

\end{itemize}

\end{frame}

%%%%%%%%%%%%%%%%%%%%%%%%%%%%%%%%%%%%

\subsection{CLT only applies when independence and sample size/skew conditions are met}
\label{mi3}

%%%%%%%%%%%%%%%%%%%%%%%%%%%%%%%%%%%%

\begin{frame}
\frametitle{3. CLT only applies when independence and sample size/skew conditions are met}

\begin{enumerate}

\item \hlGr{Independence:} Sampled observations must be independent. \\

$\:$ \\
This is difficult to verify, but is more likely if
\begin{itemize}
\item random sampling/assignment is used, and,
\item if sampling without replacement, $n$ $<$ 10\% of the population.
\end{itemize}

\pause

 \item \hlGr{Sample size/skew:} Either 

 \begin{itemize}
 \item the population distribution is normal or

 \item $n > 30$ and the population dist.\ is not extremely skewed, or

 \item $n >> 30$ (approx.\ gets better as $n$ increases).
 \end{itemize}

This is also difficult to verify for the population, but we can check it using the sample data, and assume that the sample mirrors the population.


\end{enumerate}

\end{frame}

%%%%%%%%%%%%%%%%%%%%%%%%%%%%%%%%%%%%

\begin{frame}
\frametitle{3. CLT only applies when independence and sample size/skew conditions are met}

Amongst other things, the central limit theorem is useful for 
\begin{itemize}
\item constructing confidence intervals and
\item conducting hypothesis tests.
\end{itemize}

\end{frame}

%%%%%%%%%%%%%%%%%%%%%%%%%%%%%%%%%%%%

\begin{frame}
\frametitle{}

\clicker{Which of the below visualizations is \emph{not} appropriate for checking the shape of the distribution of the sample, and hence the population?}

\begin{enumerate}[(a)]
\item histogram
\item boxplot
\item normal probability plot
\item \solnMult{mosaicplot}
\end{enumerate}

\end{frame}

%%%%%%%%%%%%%%%%%%%%%%%%%%%%%%%%%%%%

%\section{Exercises [time permitting]}
%
%%%%%%%%%%%%%%%%%%%%%%%%%%%%%%%%%%%%
%
%\begin{frame}
%\frametitle{}
%
%\clicker{
%{\footnotesize Four plots: Determine which plot (A, B, or C) is which. \\
%(1) At top: distribution for a population ($\mu = 60, \sigma = 18$), \\
%(2) a single random sample of 500 observations from this population, \\
%(3) a distribution of 500 sample means from random samples with size 18,  \\
%(4) a distribution of 500 sample means from random samples with size 81.}}
%
%\twocol{0.4}{0.6}{
%\includegraphics[width=\textwidth]{figures/cltSimLS/cltSimLS_pop}
%}
%{
%\vspace{-0.5cm}
%{\small
%\begin{enumerate}[(a)]
%\item (2) - B; (3) - A; (4) - C
%\item (2) - A; (3) - B; (4) - C
%\item (2) - C; (3) - A; (4) - D
%\item \solnMult{(2) - B; (3) - C; (4) - A}
%\end{enumerate}
%}
%}
%\vspace{-0.25cm}
%\includegraphics[width=0.32\textwidth]{figures/cltSimLS/cltSimLS_n81}
%\includegraphics[width=0.32\textwidth]{figures/cltSimLS/cltSimLS_samp}
%\includegraphics[width=0.32\textwidth]{figures/cltSimLS/cltSimLS_n18}
%
%\end{frame}
%
%%%%%%%%%%%%%%%%%%%%%%%%%%%%%%%%%%%%%
%
%\begin{frame}
%
%\disc{{\small A housing survey was conducted to determine the price of a typical home in Topanga, CA. The mean price of a house was roughly \$1.3 million with a standard deviation of \$300,000. There were no houses listed below \$600,000 but a few houses above \$3 million.}}
%
%\disc{{\small Would you expect most houses in Topanga to cost more or less than \$1.3 million? Hint: What is most likely the shape of this distribution?}}
%
%\pause
%
%\soln{Since the distribution is probably right skewed, the median would be less than the mean, and a majority of observations would be lower than the mean.}
%
%
%\end{frame}
%
%%%%%%%%%%%%%%%%%%%%%%%%%%%%%%%%%%%%%
%
%\begin{frame}
%
%\disc{{\small A housing survey was conducted to determine the price of a typical home in Topanga, CA. The mean price of a house was roughly \$1.3 million with a standard deviation of \$300,000. There were no houses listed below \$600,000 but a few houses above \$3 million.}}
%
%\clicker{Can we estimate the probability that a randomly chosen house in Topanga costs more than \$1.4 million using the normal distribution?}
%
%\begin{enumerate}[(a)]
%\item yes
%\item \solnMult{no}
%\end{enumerate}
%
%\end{frame}
%
%%%%%%%%%%%%%%%%%%%%%%%%%%%%%%%%%%%%
%
%\begin{frame}
%
%\disc{{\small A housing survey was conducted to determine the price of a typical home in Topanga, CA. The mean price of a house was roughly \$1.3 million with a standard deviation of \$300,000. There were no houses listed below \$600,000 but a few houses above \$3 million.}}
%
%\clicker{Can we estimate the probability that the mean of 60 randomly chosen houses in Topanga is more than \$1.4 million?}
%
%\begin{enumerate}[(a)]
%\item \solnMult{yes}
%\item no
%\end{enumerate}
%
%\end{frame}
%
%%%%%%%%%%%%%%%%%%%%%%%%%%%%%%%%%%%%%
%
%\begin{frame}
%
%\disc{{\small A housing survey was conducted to determine the price of a typical home in Topanga, CA. The mean price of a house was roughly \$1.3 million with a standard deviation of \$300,000. There were no houses listed below \$600,000 but a few houses above \$3 million.}}
%
%\disc{{\small What is the probability that the mean of 60 randomly chosen houses in Topanga is more than \$1.4 million?}}
%
%\pause
%
%In order to calculate $P(\bar{X} > 1.4~mil)$, we need to first determine the distribution of $\bar{X}$. According to the CLT,
%
%\pause
%
%\[ \bar{X} \pause \sim N \pause \left( mean = 1.3, \pause SE = \frac{0.3}{\sqrt{60}} = 0.0387 \right) \]
%
%\pause
%
%\begin{eqnarray*}
%P(\bar{X} > 1.4 ) &=& P\left(Z > \frac{1.4 - 1.3}{0.0387}\right) \\
%\pause
%&=& P(Z > 2.58) \\
%\pause
%&=& 1 - 0.9951 \pause =  0.0049
%\end{eqnarray*}
%
%\end{frame}
%
%%%%%%%%%%%%%%%%%%%%%%%%%%%%%%%%%%%%%

\subsection{Statistical inference methods based on the CLT depend on the same conditions as the CLT}
\label{mi4}

%%%%%%%%%%%%%%%%%%%%%%%%%%%%%%%%%%%%

\begin{frame}
\frametitle{4. Statistical inference methods based on the CLT depend on the same conditions as the CLT}

Always check these in context of the data and the research question!

\begin{enumerate}

\item \hlGr{Independence:} Sampled observations must be independent.
$\:$ \\
This is difficult to verify, but is more likely if
\begin{itemize}
\item random sampling/assignment is used, and,
\item if sampling without replacement, $n$ $<$ 10\% of the population.
\end{itemize}

\item \hlGr{Sample size/skew:} Population must be normal or sample size must be large.

\end{enumerate}

\end{frame}

%%%%%%%%%%%%%%%%%%%%%%%%%%%%%%%%%%%%

\subsection{Use confidence intervals to estimate population parameters}
\label{mi5}

%%%%%%%%%%%%%%%%%%%%%%%%%%%%%%%%%%%%

\begin{frame}
\frametitle{5. Use confidence intervals to estimate population parameters}

\vfill

\[ CI: point~estimate \pm margin~of~error \]
$\:$ \\

\pause

If the parameter of interest is the population mean, and the point estimate is the sample mean,
\[ \bar{x} \pm Z^\star \frac{s}{\sqrt{n}} \]

\vfill

\end{frame}

%%%%%%%%%%%%%%%%%%%%%%%%%%%%%%%%%%%%

\begin{frame}

\vfill

\app{3.1 Confidence interval for a single mean}{See course website for details.}

\vfill

\end{frame}

%%%%%%%%%%%%%%%%%%%%%%%%%%%%%%%%%%%%

\subsection{Critical value depends on the confidence level}
\label{mi6}

%%%%%%%%%%%%%%%%%%%%%%%%%%%%%%%%%%%%

\begin{frame}
\frametitle{6. Critical value depends on the confidence level}

\clicker{What is the critical value ($Z^\star$) for a confidence interval at the 91\% confidence level?}

\twocol{0.4}{0.6}{
\begin{enumerate}[(a)]
\item $Z^\star = 1.34$
\item $Z^\star = 1.65$
\item \solnMult{$Z^\star = 1.70$}
\item $Z^\star = 1.96$
\item $Z^\star = 2.33$
\end{enumerate}
}
{
\pause
\soln{
\includegraphics[width=\textwidth]{figures/conf_level/conf_level}
}
}

\end{frame}

%%%%%%%%%%%%%%%%%%%%%%%%%%%%%%%%%%%%

\begin{frame}
\frametitle{Common misconceptions about confidence intervals}

\begin{enumerate}

\item \textcolor{gray}{The confidence level of a confidence interval is the probability that \textbf{the specific confidence interval you construct with data from a single sample} contains the true population parameter.} \\
\textit{The confidence level is equal to the proportion of random samples that result in confidence intervals that contain the true population parameter.} \\
$\:$ \\

\pause

\item  \textcolor{gray}{A narrower confidence interval is always better.}\\
\textit{This is incorrect since the width is a function of both the confidence level and the standard error.} \\
$\:$ \\

\pause

\item   \textcolor{gray}{A wider interval means less confidence.} \\
\textit{This is incorrect since it is possible to make very precise statements with very little confidence.} \\

\end{enumerate}

\end{frame}

%%%%%%%%%%%%%%%%%%%%%%%%%%%%%%%%%%%%

\subsection{Use hypothesis tests to make decisions about population parameters}
\label{mi7}

%%%%%%%%%%%%%%%%%%%%%%%%%%%%%%%%%%%%

\begin{frame}
\frametitle{7. Use hypothesis tests to make decisions about population parameters}

Hypothesis testing framework:

\begin{enumerate}

\item Set the hypotheses.

\item Check assumptions and conditions.

\item Calculate a \hl{test statistic} and a p-value.

\item Make a decision, and interpret it in context of the research question.

\end{enumerate}

\end{frame}

%%%%%%%%%%%%%%%%%%%%%%%%%%%%%%%%%%%

\begin{frame}
\frametitle{Hypothesis testing for a population mean}

\begin{enumerate}

\item Set the hypotheses
\begin{itemize}
\item $H_0: \mu = null~value$
\item $H_A: \mu <$ or $>$ or $\ne null~value$
\end{itemize}

\pause

\item Check assumptions and conditions
\begin{itemize}
\item Independence: random sample/assignment, 10\% condition when sampling without replacement
\item Sample size / skew: $n \ge 30$ (or larger if sample is skewed), no extreme skew
\end{itemize}

\pause

\item Calculate a \hl{test statistic} and a p-value (draw a picture!)
\[ Z = \frac{\bar{x} - \mu}{SE},~where~SE = \frac{s}{\sqrt{n}} \]

\pause

\item Make a decision, and interpret it in context of the research question
\begin{itemize}
\item If p-value $< \alpha$, reject $H_0$, data provide evidence for $H_A$
\item If p-value $> \alpha$, do not reject $H_0$, data do not provide evidence for $H_A$
\end{itemize}

\end{enumerate}

\end{frame}

%%%%%%%%%%%%%%%%%%%%%%%%%%%%%%%%%%%%

\begin{frame}
\frametitle{}

\vfill

\app{3.2 Hypothesis testing for a single mean}{See course website for details.}

\vfill

\end{frame}

%%%%%%%%%%%%%%%%%%%%%%%%%%%%%%%%%%%%

\begin{frame}

\clicker{Which of the following is the correct interpretation of the p-value from App Ex 3.2?}

\begin{enumerate}[(a)]
\item The probability that average GPA of Duke students has changed since 2001.
\item The probability that average GPA of Duke students has not changed since 2001.
\item The probability that average GPA of Duke students has not changed since 2001, if in fact a random sample of 63 Duke students this year have an average GPA of 3.58 or higher.
\item The probability that a random sample of 63 Duke students have an average GPA of 3.58 or higher, if in fact the average GPA has not changed since 2001.
\item \solnMult{The probability that a random sample of 63 Duke students have an average GPA of 3.58 or higher or 3.16 or lower, if in fact the average GPA has not changed since 2001.}
\end{enumerate}

\end{frame}

%%%%%%%%%%%%%%%%%%%%%%%%%%%%%%%%%%%%

\begin{frame}
\frametitle{Common misconceptions about hypothesis testing}

\begin{enumerate}

\item \textcolor{gray}{P-value is the probability that the null hypothesis is true} \\
\textit{A p-value is the probability of getting a sample that results in a test statistic as or more extreme than what you actually observed (in the direction of $H_A$, if in fact $H_0$ is correct. It is a conditional probability, conditioned on $H_0$ being correct.} \\
$\:$ \\

\pause

\item  \textcolor{gray}{A high p-value confirms the null hypothesis.}\\
\textit{A high p-value means the data do not provide convincing evidence for $H_A$ and hence that $H_0$ can't be rejected.} \\
$\:$ \\

\pause

\item   \textcolor{gray}{A low p-value confirms the alternative hypothesis.} \\
\textit{A low p-value means the data provide convincing evidence for $H_A$, but not necessarily that it is confirmed.} \\

\end{enumerate}

\end{frame}

%%%%%%%%%%%%%%%%%%%%%%%%%%%%%%%%%%%%

\section{Summary}

%%%%%%%%%%%%%%%%%%%%%%%%%%%%%%%%%%%%

\begin{frame}
\frametitle{Summary of main ideas}

\vfill

\begin{enumerate}

\item \nameref{mi1}

\item \nameref{mi2}

\item \nameref{mi3}

\item \nameref{mi4}

\item \nameref{mi5}

\item \nameref{mi6}

\item \nameref{mi7}

\end{enumerate}

\vfill

\end{frame}

%%%%%%%%%%%%%%%%%%%%%%%%%%%%%%%%%%%

\end{document}