% -*- TeX-engine: xetex; eval: (auto-fill-mode 0); eval: (visual-line-mode 1); -*-
% Compile with XeLaTeX

%%%%%%%%%%%%%%%%%%%%%%%
% Option 1: Slides: (comment for handouts)   %
%%%%%%%%%%%%%%%%%%%%%%%

%\documentclass[slidestop,compress,mathserif,12pt,t,professionalfonts,xcolor=table]{beamer}
%
%% solution stuff
%\newcommand{\solnMult}[1]{
%\only<1>{#1}
%\only<2->{\red{\textbf{#1}}}
%}
%\newcommand{\soln}[1]{\textit{#1}}

%%%%%%%%%%%%%%%%%%%%%%%%%%%%%%%
% Option 2: Handouts, without solutions (post before class)    %
%%%%%%%%%%%%%%%%%%%%%%%%%%%%%%%

 \documentclass[11pt,containsverbatim,handout,xcolor=xelatex,dvipsnames,table]{beamer}

 % handout layout
 \usepackage{pgfpages}
 \pgfpagesuselayout{4 on 1}[letterpaper,landscape,border shrink=5mm]

 % solution stuff
 \newcommand{\solnMult}[1]{#1}
 \newcommand{\soln}[1]{}

%%%%%%%%%%%%%%%%%%%%%%%%%%%%%%%%%%%%
% Option 3: Handouts, with solutions (may post after class if need be)    %
%%%%%%%%%%%%%%%%%%%%%%%%%%%%%%%%%%%%

% \documentclass[11pt,containsverbatim,handout,xcolor=xelatex,dvipsnames,table]{beamer}

% % handout layout
% \usepackage{pgfpages}
% \pgfpagesuselayout{4 on 1}[letterpaper,landscape,border shrink=5mm]

% % solution stuff
% \newcommand{\solnMult}[1]{\red{\textbf{#1}}}
% \newcommand{\soln}[1]{\textit{#1}}

%%%%%%%%%%%%%%%%%%%%%%%%%%%%%%%
% Option 4: Notes Only
%%%%%%%%%%%%%%%%%%%%%%%%%%%%%%%

% % See http://tex.stackexchange.com/questions/114219/add-notes-to-latex-beamer
% \documentclass[10pt,containsverbatim,xcolor=xelatex,dvipsnames,table,notes=only]{beamer}

% % handout layout
% \usepackage{pgfpages}
% \pgfpagesuselayout{2 on 1}[letterpaper, landscape, border shrink=5mm]

% % solution stuff
% \newcommand{\solnMult}[1]{#1}
% \newcommand{\soln}[1]{}

%%%%%%%%%%
% Load style file, defaults  %
%%%%%%%%%%

\input{../../lec_style.tex}
% You cannot use numbers when defining variables.  Hence the use of letters, A, B, C, etc.

% Personal Info
\newcommand{\FirstName}{Mine}
\newcommand{\LastName}{\c{C}etinkaya-Rundel}
\newcommand{\OfficeHours}{Generally TR 12:30 - 1:30pm}

% Electronic Info
\newcommand{\PersonalSite}{http://stat.duke.edu/~mc301}
\newcommand{\CourseSite}{http://bit.ly/sta104su15}
\newcommand{\Email}{mine@stat.duke.edu}

% TAs
\newcommand{\TAA}{Andrew Wong}

% Exam Dates
\newcommand{\ExamDate}{May 29, 11am - 12:30pm (in class)}
\newcommand{\FinalDate}{June 24, 11am - 2pm}

% ALT ALT
% \input{../../definitions_custom.tex}

%%%%%%%%%%%
% Cover slide info    %
%%%%%%%%%%%

\title{Unit 4: Inference for numerical data}
\subtitle{1. Decision errors, significance levels, sample size \& power}
\author{Sta 104 - Summer 2015}
\date{June 1, 2015}
\institute{Duke University, Department of Statistical Science}

%%%%%%%%%%%%%%%%%%%%%%%%%
% Begin document and set Helvetica Neue font   %
%%%%%%%%%%%%%%%%%%%%%%%%%

\begin{document}
\fontspec[Ligatures=TeX]{Helvetica Neue Light}

%%%%%%%%%%%%%%%%%%%%%%%%%%%%%%%%%%%

% Title Page

\begin{frame}[plain]

\titlepage
\vfill
{\scriptsize \webLink{\PersonalSite}{Dr. \LastName{}} \hfill Slides posted at  \webLink{\CourseSite}{\CourseSite}}
\addtocounter{framenumber}{-1} 

\end{frame}

%%%%%%%%%%%%%%%%%%%%%%%%%%%%%%%%%%%

\section{Housekeeping}

%%%%%%%%%%%%%%%%%%%%%%%%%%%%%%%%%%%

\begin{frame}
\frametitle{Announcements}

\begin{itemize}

\item PS3 due tonight

\item Project proposals due Thursday night

\item MT corrections extra credit: Work \textbf{as a team} to write up a collective exam corrections document that discusses all questions missed by any member of the team. Your corrections should show full work and explain reasoning, even for the multiple choice questions. Due by the end of the day on Wednesday, June 3. \textbf{Extra credit:} +2 points on the exam.

\end{itemize}

\end{frame}

%%%%%%%%%%%%%%%%%%%%%%%%%%%%%%%%%%%

\section{Main ideas}

%%%%%%%%%%%%%%%%%%%%%%%%%%%%%%%%%%%%

\subsection{Hypothesis tests and confidence intervals at equivalent significance/confidence levels should agree}
\label{mi1dec}

%%%%%%%%%%%%%%%%%%%%%%%%%%%%%%%%%%%%

\begin{frame}
\frametitle{1. Hypothesis tests and confidence intervals at equivalent significance/confidence levels should agree}

\twocol{0.5}{0.5}{
\begin{center}
Two sided\\
~\\
\includegraphics[width=\textwidth]{figures/sig_conf_equiv/CL95_twosided} \\
95\% confidence level \\
is equivalent to \\
two sided HT with $\alpha = 0.05$
\end{center}
}
{\pause
\begin{center}
One sided\\
~\\
\includegraphics[width=\textwidth]{figures/sig_conf_equiv/CL95_onesided} \\
95\% confidence level \\
is equivalent to \\
one sided HT with $\alpha = 0.025$
\end{center}
}

\end{frame}

%%%%%%%%%%%%%%%%%%%%%%%%%%%%%%%%%%%%

\begin{frame}
\frametitle{}

\clicker{What is the significance level of a two-sided hypothesis test that is equivalent to a 90\% confidence interval? \textit{Hint: Draw a picture and mark the confidence level in the center.}}

\begin{enumerate}[(a)]
\item 0.001
\item 0.01
\item 0.025
\item 0.05
\item \solnMult{0.10}
\end{enumerate}

\end{frame}

%%%%%%%%%%%%%%%%%%%%%%%%%%%%%%%%%%%%

\begin{frame}
\frametitle{}

\clicker{What is the significance level of a one-sided hypothesis test that is equivalent to a 90\% confidence interval? \textit{Hint: Draw a picture and mark the confidence level in the center.}}

\begin{enumerate}[(a)]
\item 0.001
\item 0.01
\item 0.025
\item \solnMult{0.05}
\item 0.10
\end{enumerate}

\end{frame}

%%%%%%%%%%%%%%%%%%%%%%%%%%%%%%%%%%%%

\begin{frame}
\frametitle{}

\clicker{What is the confidence level of a confidence interval that is equivalent to a two-sided hypothesis test with $\alpha = 0.01$. \textit{Hint: Draw a picture and mark the confidence level in the center.}}

\begin{enumerate}[(a)]
\item 0.80
\item 0.90
\item 0.95
\item 0.98
\item \solnMult{0.99}
\end{enumerate}

\end{frame}

%%%%%%%%%%%%%%%%%%%%%%%%%%%%%%%%%%%%

\begin{frame}
\frametitle{}

\clicker{What is the confidence level of a confidence interval that is equivalent to a one-sided hypothesis test with $\alpha = 0.01$. \textit{Hint: Draw a picture and mark the confidence level in the center.}}

\begin{enumerate}[(a)]
\item 0.80
\item 0.90
\item 0.95
\item \solnMult{0.98}
\item 0.99
\end{enumerate}

\end{frame}

%%%%%%%%%%%%%%%%%%%%%%%%%%%%%%%%%%%%

\begin{frame}
\frametitle{}

\clicker{A 95\% confidence interval for the average normal body temperature of humans is found to be (98.1 F, 98.4 F). Which of the following is \emph{true}?}

\begin{enumerate}[(a)]
\item The hypothesis $H_0: \mu = 98.2$ would be rejected at $\alpha = 0.05$ in favor of $H_A: \mu \ne 98.2$.
\item The hypothesis $H_0: \mu = 98.2$ would be rejected at $\alpha = 0.025$ in favor of $H_A: \mu > 98.2$.
\item \solnMult{The hypothesis $H_0: \mu = 98$ would be rejected using a 90\% confidence interval.}
\item The hypothesis $H_0: \mu = 98.2$ would be rejected using a 99\% confidence interval.
\end{enumerate}

\end{frame}

%%%%%%%%%%%%%%%%%%%%%%%%%%%%%%%%%%%%

\subsection{Results that are statistically significant are not necessarily practically significant}
\label{mi2dec}

%%%%%%%%%%%%%%%%%%%%%%%%%%%%%%%%%%%%

\begin{frame}
\frametitle{2. Results that are statistically significant are not necessarily practically significant}

\clicker{All else held equal, will p-value be lower if $n = 100$ or $n = 10,000$?}

\begin{enumerate}[(a)]
\item $n = 100$
\item \solnMult{$n = 10,000$}
\end{enumerate}

\soln{\pause \pause
Suppose $\bar{x} = 5$, $s = 2$, $H_0: \mu = 4.5$, and $H_A: \mu \ge 4.5$.\\
\pause
{\small
\begin{eqnarray*}
Z_{n = 100} &=& \frac{5 - 4.5}{\frac{2}{\sqrt{100}}} \pause = \frac{5 - 4.5}{\frac{2}{10}} = \frac{0.5}{0.2} = 2.5,~~~\text{p-value} = 0.0062 \\
\pause
Z_{n = 10000} &=& \frac{5 - 4.5}{\frac{2}{\sqrt{10000}}} \pause = \frac{5 - 4.5}{\frac{2}{100}} = \frac{0.5}{0.02} = 25,~~~\text{p-value} \approx 0
\end{eqnarray*}
}
\pause
\begin{center}
As $n$ increases - $SE$ $\downarrow$, $Z$ $\uparrow$, p-value $\downarrow$
\end{center}
}

\end{frame}

%%%%%%%%%%%%%%%%%%%%%%%%%%%%%%%%%%%%

\subsection{Calculate the sample size a priori to achieve desired margin of error}
\label{mi3dec}

%%%%%%%%%%%%%%%%%%%%%%%%%%%%%%%%%%%%

\begin{frame}
\frametitle{3. Calculate the sample size \textit{a priori} to achieve desired margin of error}

\vfill

\app{4.1 Sample size}{See course website for details.}

\vfill

\end{frame}

%%%%%%%%%%%%%%%%%%%%%%%%%%%%%%%%%%%%

\subsection{Hypothesis tests are prone to decision errors}
\label{mi4dec}

%%%%%%%%%%%%%%%%%%%%%%%%%%%%%%%%%%%%

\begin{frame}
\frametitle{4. Hypothesis tests are prone to decision errors}

\begin{center}
\begin{tabular}{l l | c c}
\multicolumn{2}{c}{} & \multicolumn{2}{c}{\textbf{Decision}} \\
& & fail to reject $H_0$ &  reject $H_0$ \\
  \cline{2-4}
& $H_0$ true & \onslide<3->{\green{$\checkmark$}} &  \onslide<4->{\red{Type 1 Error, $\alpha$}} \\
\raisebox{1.5ex}{\textbf{Truth}} & $H_A$ true & \onslide<5->{\red{Type 2 Error, $\beta$}} & \onslide<6->{\green{Power, $1 - \beta$}} \\
  \cline{2-4}
\end{tabular}
\end{center}

\begin{itemize}
\item \onslide<4->{A \hl{Type 1 Error} is rejecting the null hypothesis when $H_0$ is true: $\alpha$
\begin{itemize}
\item For those cases where $H_0$ is actually true, we do not want to incorrectly reject it more than 5\% of those times
\item Increasing $\alpha$ increases the Type 1 error rate, hence we prefer to small values of $\alpha$
\end{itemize}
}

\item \onslide<5->{A \hl{Type 2 Error} is failing to reject the null hypothesis when $H_A$ is true: $\beta$}

\item \onslide<6->{\hl{Power} is the probability of correctly rejecting $H_0$, and hence the complement of the probability of a Type 2 Error: $1 - \beta$}

\end{itemize}

\end{frame}

%%%%%%%%%%%%%%%%%%%%%%%%%%%%%%%%%%%%

\subsection{Power depends on the effect size, $\alpha$, $n$, and $s$}
\label{mi5dec}

%%%%%%%%%%%%%%%%%%%%%%%%%%%%%%%%%%%%

\begin{frame}
\frametitle{5. Power depends on the $n$, $a$, $\alpha$, effect size}

Power can be increased (and hence Type 2 error rate can be decreased) by

\pause

\begin{itemize}

\item increasing the sample size

\pause

\item decreasing the standard deviation of the sample (difficult to ensure but cautious measurement process and limiting the population so that it is more homogenous may help)

\pause

\item increasing $\alpha$

\pause

\item increasing the \hl{effect size}

\end{itemize}

\end{frame}

%%%%%%%%%%%%%%%%%%%%%%%%%%%%%%%%%%%%

\section{Summary}

%%%%%%%%%%%%%%%%%%%%%%%%%%%%%%%%%%%%

\begin{frame}
\frametitle{Summary of main ideas}

\vfill

\begin{enumerate}

\item \nameref{mi1dec}

\item \nameref{mi2dec}

\item \nameref{mi3dec}

\item \nameref{mi4dec}

\item \nameref{mi5dec}

\end{enumerate}

\vfill

\end{frame}

%%%%%%%%%%%%%%%%%%%%%%%%%%%%%%%%%%%

\end{document}