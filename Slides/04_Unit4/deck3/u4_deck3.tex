% -*- TeX-engine: xetex; eval: (auto-fill-mode 0); eval: (visual-line-mode 1); -*-
% Compile with XeLaTeX

%%%%%%%%%%%%%%%%%%%%%%%
% Option 1: Slides: (comment for handouts)   %
%%%%%%%%%%%%%%%%%%%%%%%

\documentclass[slidestop,compress,mathserif,12pt,t,professionalfonts,xcolor=table]{beamer}

% solution stuff
\newcommand{\solnMult}[1]{
\only<1>{#1}
\only<2->{\red{\textbf{#1}}}
}
\newcommand{\soln}[1]{\textit{#1}}

%%%%%%%%%%%%%%%%%%%%%%%%%%%%%%%
% Option 2: Handouts, without solutions (post before class)    %
%%%%%%%%%%%%%%%%%%%%%%%%%%%%%%%

% \documentclass[11pt,containsverbatim,handout,xcolor=xelatex,dvipsnames,table]{beamer}

% % handout layout
% \usepackage{pgfpages}
% \pgfpagesuselayout{4 on 1}[letterpaper,landscape,border shrink=5mm]

% % solution stuff
% \newcommand{\solnMult}[1]{#1}
% \newcommand{\soln}[1]{}

%%%%%%%%%%%%%%%%%%%%%%%%%%%%%%%%%%%%
% Option 3: Handouts, with solutions (may post after class if need be)    %
%%%%%%%%%%%%%%%%%%%%%%%%%%%%%%%%%%%%

% \documentclass[11pt,containsverbatim,handout,xcolor=xelatex,dvipsnames,table]{beamer}

% % handout layout
% \usepackage{pgfpages}
% \pgfpagesuselayout{4 on 1}[letterpaper,landscape,border shrink=5mm]

% % solution stuff
% \newcommand{\solnMult}[1]{\red{\textbf{#1}}}
% \newcommand{\soln}[1]{\textit{#1}}

%%%%%%%%%%%%%%%%%%%%%%%%%%%%%%%
% Option 4: Notes Only
%%%%%%%%%%%%%%%%%%%%%%%%%%%%%%%

% % See http://tex.stackexchange.com/questions/114219/add-notes-to-latex-beamer
% \documentclass[10pt,containsverbatim,xcolor=xelatex,dvipsnames,table,notes=only]{beamer}

% % handout layout
% \usepackage{pgfpages}
% \pgfpagesuselayout{2 on 1}[letterpaper, landscape, border shrink=5mm]

% % solution stuff
% \newcommand{\solnMult}[1]{#1}
% \newcommand{\soln}[1]{}

%%%%%%%%%%
% Load style file, defaults  %
%%%%%%%%%%

\input{../../lec_style.tex}
% You cannot use numbers when defining variables.  Hence the use of letters, A, B, C, etc.

% Personal Info
\newcommand{\FirstName}{Mine}
\newcommand{\LastName}{\c{C}etinkaya-Rundel}
\newcommand{\OfficeHours}{Generally TR 12:30 - 1:30pm}

% Electronic Info
\newcommand{\PersonalSite}{http://stat.duke.edu/~mc301}
\newcommand{\CourseSite}{http://bit.ly/sta104su15}
\newcommand{\Email}{mine@stat.duke.edu}

% TAs
\newcommand{\TAA}{Andrew Wong}

% Exam Dates
\newcommand{\ExamDate}{May 29, 11am - 12:30pm (in class)}
\newcommand{\FinalDate}{June 24, 11am - 2pm}

% ALT ALT
% \input{../../definitions_custom.tex}

%%%%%%%%%%%
% Cover slide info    %
%%%%%%%%%%%

\title{Unit 4: Inference for numerical data}
\subtitle{3. ANOVA}
\author{Sta 104 - Summer 2015}
\date{June 3, 2015}
\institute{Duke University, Department of Statistical Science}

%%%%%%%%%%%%%%%%%%%%%%%%%
% Begin document and set Helvetica Neue font   %
%%%%%%%%%%%%%%%%%%%%%%%%%

\begin{document}
\fontspec[Ligatures=TeX]{Helvetica Neue Light}

%%%%%%%%%%%%%%%%%%%%%%%%%%%%%%%%%%%

% Title Page

\begin{frame}[plain]

\titlepage
\vfill
{\scriptsize \webLink{\PersonalSite}{Dr. \LastName{}} \hfill Slides posted at  \webLink{\CourseSite}{\CourseSite}}
\addtocounter{framenumber}{-1} 

\end{frame}

%%%%%%%%%%%%%%%%%%%%%%%%%%%%%%%%%%%

\section{Housekeeping}

%%%%%%%%%%%%%%%%%%%%%%%%%%%%%%%%%%%

\begin{frame}
\frametitle{Announcements}

\begin{itemize}

\item 

\end{itemize}

\end{frame}

%%%%%%%%%%%%%%%%%%%%%%%%%%%%%%%%%%%

\section{Main ideas}

%%%%%%%%%%%%%%%%%%%%%%%%%%%%%%%%%%%

\subsection{It is difficult to simultaneously compare many groups}
\label{mi1}

%%%%%%%%%%%%%%%%%%%%%%%%%%%%%%%%%%%

\begin{frame}
\frametitle{1. It is difficult to simultaneously compare many groups}

\twocol{0.35}{0.65}
{
\begin{center}
\includegraphics[width=\textwidth]{figures/jelly_beans}
\end{center}
}
{
\pause
\disc{How would you check this rumor? Imagine that doctors can assign an ``acne score" to 
patients on a 0 - 100 scale.
\begin{itemize}
\item What is the research question?
\item How would you conduct your study?
\item What statistical test would you use?
\end{itemize}
}
}

\hfill \\

\pause

\soln{
Use an independent samples t-test:
\[ H_0: \mu_{\textmd{green jelly beans}} = \mu_{\textmd{placebo}} \]
\[ H_A: \mu_{\textmd{green jelly beans}} \neq \mu_{\textmd{placebo}} \]
}

\end{frame}
	
%%%%%%%%%%%%%%%%%%%%%%%%%%%%%%%%%%%

\begin{frame}
\frametitle{}

\vfill
  
\begin{center}
\url{http://imgs.xkcd.com/comics/significant.png}
\end{center}

\end{frame}

%%%%%%%%%%%%%%%%%%%%%%%%%%%%%%%%%%%%

\begin{frame}
\frametitle{1. It is difficult to simultaneously compare many groups}

\textbf{Suppose} $\alpha = 0.05$.

\disc{What is the probability of rejecting the following null hypothesis 
when in fact it is true?
\[ H_0: \mu_{\textmd{purple}} = \mu_{\textmd{placebo}} \]
}

$\:$ \\

\pause

\soln{P(Type 1 Error) = 0.05}

\end{frame}

%%%%%%%%%%%%%%%%%%%%%%%%%%%%%%%%%%%%

\begin{frame}
\frametitle{1. It is difficult to simultaneously compare many groups}

\textbf{Suppose} $\alpha = 0.05$.

\clicker{If all the tests are independent and if no color of Jelly bean has any link to acne, what is the
probability of making at least one type I error in the 20 trials?}
\begin{enumerate}[(a)]
\item 5\%
\item 36\%
\item \solnMult{64\%} \only<2>{\soln{\red{$\rightarrow$ $1 - 0.95^{20} \approx 0.64$}}}
\item 95\%
\end{enumerate}

\end{frame}

%%%%%%%%%%%%%%%%%%%%%%%%%%%%%%%%%%%%

\subsection{ANOVA is useful for testing if there is \emph{some} difference
between the means of many different groups}
\label{mi2}

%%%%%%%%%%%%%%%%%%%%%%%%%%%%%%%%%%%

\begin{frame}
\frametitle{2. ANOVA is useful for testing if there is \emph{some} difference
between the means of many different groups.}

Null hypothesis for $F$-test (the test associated with ANOVA):
\[ H_0: \mu_{\textmd{placebo}} = \mu_{\textmd{purple}} = \mu_{\textmd{brown}} =
\ldots = \mu_{\textmd{peach}} = \mu_{\textmd{orange}} \]

\end{frame}

%%%%%%%%%%%%%%%%%%%%%%%%%%%%%%%%%%%%

\begin{frame}
\frametitle{}

\clicker{Which of the following is a correct version of the alternative
hypothesis?}

\begin{enumerate}[(a)]
\item For any two groups, including the placebo group, no two group means are the same.
\item For any two groups, not including the placebo group, no two group means are the
same.
\item Amongst the jelly bean groups, there are at least two groups that have different group means.
\item \solnMult{Amongst all groups, there are at least two groups that have different group means.}
\end{enumerate}

\pause

The practical implication of this alternative is: ``At least one color of
jelly bean is linked to acne.''

\end{frame}

%%%%%%%%%%%%%%%%%%%%%%%%%%%%%%%%%%%%

\subsection{The test compares between group variation to within group variation}
\label{mi3}

%%%%%%%%%%%%%%%%%%%%%%%%%%%%%%%%%%%

\begin{frame}
\frametitle{3. The test compares between group variation to within group variation}

\[ \sum {\color{blue} \mid}^2 \Big/ \sum {\color{green} \mid}^2 \]

\begin{center}
\includegraphics[scale=0.6]{figures/anova-middle-ground-jelly-bean}
\end{center}

\end{frame}

%%%%%%%%%%%%%%%%%%%%%%%%%%%%%%%%%%%

\begin{frame}
\frametitle{Relatively large WITHIN group variation: little apparent difference}

\[ \sum {\color{blue} \mid}^2 \Big/ \sum {\color{green} \mid}^2 \]

\begin{center}
\includegraphics[scale=0.6]{figures/anova-lots-of-within-group-jelly-bean}
\end{center}

\end{frame}

%%%%%%%%%%%%%%%%%%%%%%%%%%%%%%%%%%%

\begin{frame}
\frametitle{Relatively large BETWEEN group variation: there may be a difference}

\[ \sum {\color{blue} \mid}^2 \Big/ \sum {\color{green} \mid}^2 \]

\begin{center}
\includegraphics[scale=0.6]{figures/anova-lots-of-between-group-jelly-bean}
\end{center}

\end{frame}

%%%%%%%%%%%%%%%%%%%%%%%%%%%%%%%%%%%

\begin{frame}
\frametitle{3. The F-test is based on comparing between group to within group variation}

For historical reasons, we use a modification of this ratio called the $F$-statistic
\[ F = \frac{\sum {\color{blue} \mid}^2 ~ / ~ (k - 1)}{\sum {\color{green} \mid}^2 ~ / ~ (n - k)} ~ =  ~ \frac{MSG}{MSE} \]
where $k$ is the \# of groups and $n$ is the \# of observations

\pause

{\small
\begin{center}
\begin{tabular}{lllrrr}
\hline
		& Df  	& Sum Sq 								& Mean Sq	& F value 		& Pr($>$F) \\ 
\hline
Between 	& k - 1 	& $\sum {\color{blue}\mid}^2$  					& MSG 		& $F_{obs}$ 	& $p-value$ \\ 
Within  	& n - k 	& $\sum {\color{green}\mid}^2$ 				& MSE 		&		 	&  \\ 
\hline
Total		& n - 1 	& $\sum ({\color{blue}\mid} + {\color{green}\mid})^2$	&                	&                	&
\end{tabular}
\end{center}
}

\end{frame}

%%%%%%%%%%%%%%%%%%%%%%%%%%%%%%%%%%%

\begin{frame}
\frametitle{}

\clicker{The p-value for the $F$-test is 0.045, and $\alpha = 0.05$. 
What is the most accurate statement of the results?}

\begin{enumerate}[(a)]
\item At least one color of jelly bean is linked to acne.
\item At least one color of jelly bean is not linked to acne.
\item \solnMult{There is little evidence that any color of jelly bean is linked to acne.}
\item Jelly beans definitely do not cause acne.
\end{enumerate}

\end{frame}

%%%%%%%%%%%%%%%%%%%%%%%%%%%%%%%%%%%

\begin{frame}
\frametitle{}

\clicker{For the $F$-test with $\alpha = 0.05$, what is the probability 
of incorrectly rejecting the null?}

\begin{enumerate}[(a)]
\item \solnMult{5\%}
\item 36\%
\item 64\%
\item 95\%
\end{enumerate}

\end{frame}

%%%%%%%%%%%%%%%%%%%%%%%%%%%%%%%%%%%

\begin{frame}

\vfill

\app{4.5 ANOVA - Pt 1}{See the course webpage for details.}

\vfill

\end{frame}

%%%%%%%%%%%%%%%%%%%%%%%%%%%%%%%%%%

\section{Summary}

%%%%%%%%%%%%%%%%%%%%%%%%%%%%%%%%%%%%

\begin{frame}
\frametitle{Summary of main ideas}

\vfill

\begin{enumerate}

\item \nameref{mi1}

\item \nameref{mi2}

\item \nameref{mi3}

\end{enumerate}

\vfill

\end{frame}

%%%%%%%%%%%%%%%%%%%%%%%%%%%%%%%%%%%

\end{document}












