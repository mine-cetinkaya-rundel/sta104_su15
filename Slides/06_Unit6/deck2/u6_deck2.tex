% -*- TeX-engine: xetex; eval: (auto-fill-mode 0); eval: (visual-line-mode 1); -*-
% Compile with XeLaTeX

%%%%%%%%%%%%%%%%%%%%%%%
% Option 1: Slides: (comment for handouts)   %
%%%%%%%%%%%%%%%%%%%%%%%

\documentclass[slidestop,compress,mathserif,12pt,t,professionalfonts,xcolor=table]{beamer}

% solution stuff
\newcommand{\solnMult}[1]{
\only<1>{#1}
\only<2->{\red{\textbf{#1}}}
}
\newcommand{\soln}[1]{\textit{#1}}

%%%%%%%%%%%%%%%%%%%%%%%%%%%%%%%
% Option 2: Handouts, without solutions (post before class)    %
%%%%%%%%%%%%%%%%%%%%%%%%%%%%%%%

% \documentclass[11pt,containsverbatim,handout,xcolor=xelatex,dvipsnames,table]{beamer}

% % handout layout
% \usepackage{pgfpages}
% \pgfpagesuselayout{4 on 1}[letterpaper,landscape,border shrink=5mm]

% % solution stuff
% \newcommand{\solnMult}[1]{#1}
% \newcommand{\soln}[1]{}

%%%%%%%%%%%%%%%%%%%%%%%%%%%%%%%%%%%%
% Option 3: Handouts, with solutions (may post after class if need be)    %
%%%%%%%%%%%%%%%%%%%%%%%%%%%%%%%%%%%%

% \documentclass[11pt,containsverbatim,handout,xcolor=xelatex,dvipsnames,table]{beamer}

% % handout layout
% \usepackage{pgfpages}
% \pgfpagesuselayout{4 on 1}[letterpaper,landscape,border shrink=5mm]

% % solution stuff
% \newcommand{\solnMult}[1]{\red{\textbf{#1}}}
% \newcommand{\soln}[1]{\textit{#1}}

%%%%%%%%%%%%%%%%%%%%%%%%%%%%%%%
% Option 4: Notes Only
%%%%%%%%%%%%%%%%%%%%%%%%%%%%%%%

% % See http://tex.stackexchange.com/questions/114219/add-notes-to-latex-beamer
% \documentclass[10pt,containsverbatim,xcolor=xelatex,dvipsnames,table,notes=only]{beamer}

% % handout layout
% \usepackage{pgfpages}
% \pgfpagesuselayout{2 on 1}[letterpaper, landscape, border shrink=5mm]

% % solution stuff
% \newcommand{\solnMult}[1]{#1}
% \newcommand{\soln}[1]{}

%%%%%%%%%%
% Load style file, defaults  %
%%%%%%%%%%

\input{../../lec_style.tex}
% You cannot use numbers when defining variables.  Hence the use of letters, A, B, C, etc.

% Personal Info
\newcommand{\FirstName}{Mine}
\newcommand{\LastName}{\c{C}etinkaya-Rundel}
\newcommand{\OfficeHours}{Generally TR 12:30 - 1:30pm}

% Electronic Info
\newcommand{\PersonalSite}{http://stat.duke.edu/~mc301}
\newcommand{\CourseSite}{http://bit.ly/sta104su15}
\newcommand{\Email}{mine@stat.duke.edu}

% TAs
\newcommand{\TAA}{Andrew Wong}

% Exam Dates
\newcommand{\ExamDate}{May 29, 11am - 12:30pm (in class)}
\newcommand{\FinalDate}{June 24, 11am - 2pm}

% ALT ALT
% \input{../../definitions_custom.tex}

%%%%%%%%%%%
% Cover slide info    %
%%%%%%%%%%%

\title{Unit 6: Simple linear regression}
\subtitle{2. Outliers \& Inference for SLR}
\author{Sta 104 - Summer 2015}
\date{June 15, 2015}
\institute{Duke University, Department of Statistical Science}

%%%%%%%%%%%%%%%%%%%%%%%%%
% Begin document and set Helvetica Neue font   %
%%%%%%%%%%%%%%%%%%%%%%%%%

\begin{document}
\fontspec[Ligatures=TeX]{Helvetica Neue Light}

%%%%%%%%%%%%%%%%%%%%%%%%%%%%%%%%%%%

% Title Page

\begin{frame}[plain]

\titlepage
\vfill
{\scriptsize \webLink{\PersonalSite}{Dr. \LastName{}} \hfill Slides posted at  \webLink{\CourseSite}{\CourseSite}}
\addtocounter{framenumber}{-1} 

\end{frame}

%%%%%%%%%%%%%%%%%%%%%%%%%%%%%%%%%%%

\section{Housekeeping}

%%%%%%%%%%%%%%%%%%%%%%%%%%%%%%%%%%%

\begin{frame}
\frametitle{Announcements}

\begin{itemize}

\item 

\end{itemize}

\end{frame}

%%%%%%%%%%%%%%%%%%%%%%%%%%%%%%%%%%%%

\section{Main ideas}

%%%%%%%%%%%%%%%%%%%%%%%%%%%%%%%%%%%%

\subsection{$R^2$ assesses model fit -- higher the better}
\label{mi1}

%%%%%%%%%%%%%%%%%%%%%%%%%%%%%%%%%%%%

\begin{frame}[fragile]
\frametitle{(1) $R^2$ assesses model fit -- higher the better}

\begin{itemize}

\item $R^2$: percentage of variability in $y$ explained by the model.

\pause

\item For single predictor regression: $R^2$ is the square of the correlation coefficient, $R$.
{\tiny
\begin{Verbatim}[frame=single, formatcom=\color{blue}]
cor(murder$annual_murders_per_mil, murder$perc_pov)^2
\end{Verbatim}
}
{\tiny
\begin{Verbatim}[frame=single, formatcom=\color{gray}]
[1] 0.7052275
\end{Verbatim}
}

\pause

\item For all regression: $R^2 = \frac{SS_{reg}}{SS_{tot}}$

{\tiny
\begin{Verbatim}[frame=single, formatcom=\color{blue}]
m1 = lm(annual_murders_per_mil ~ perc_pov, data = murder)
\end{Verbatim}
}

{\tiny
\begin{Verbatim}[frame=single, formatcom=\color{gray}]
Analysis of Variance Table

Response: annual_murders_per_mil
          Df  Sum Sq Mean Sq F value    Pr(>F)    
perc_pov   1 1308.34 1308.34  43.064 3.638e-06 ***
Residuals 18  546.86   30.38  
\end{Verbatim}
}

\end{itemize}

\pause

\begin{adjustwidth}{-0.25cm}{0cm}
{\small
\[ R^2 = \frac{explained~variabilty}{total~variability} \pause =\frac{SS_{reg}}{SS_{tot}} \pause = \frac{1308.34}{1308.34 + 546.86} \pause = \frac{1308.34}{1855.2} \approx 0.71 \]
}
\end{adjustwidth}

\end{frame}

%%%%%%%%%%%%%%%%%%%%%%%%%%%%%%%%%%%%

\begin{frame}
\frametitle{}

\twocol{0.65}{0.35}{
\clicker{$R^2$ for the regression model for predicting annual murders per million based on percentage living in poverty is roughly 71\%. Which of the following is the correct interpretation of this value?}
}
{
\begin{center}
\includegraphics[width=\textwidth]{figures/murder/annual_murders_per_mil_perc_pov}
\end{center}
}

$\:$ \\

\begin{enumerate}[(a)]

\item 71\% of the variability in percentage living in poverty is explained by the model.

\item 84\% of the variability in the murder rates is explained by the model, i.e. percentage living in poverty.

\item \solnMult{71\% of the variability in the murder rates is explained by the model, i.e. percentage living in poverty.}

\item 71\% of the time percentage living in poverty predicts murder rates accurately.

\end{enumerate}

\end{frame}

%%%%%%%%%%%%%%%%%%%%%%%%%%%%%%%%%%

\subsection{Inference for regression uses the T distribution}
\label{mi2}

%%%%%%%%%%%%%%%%%%%%%%%%%%%%%%%%%%%%

\begin{frame}
\frametitle{Inference for regression uses the T distribution}

\begin{itemize}

\item Use a T distribution for inference on the slope, with degrees of freedom $n - 2$
\begin{itemize}
\item Degrees of freedom for the slope(s) in regression is $df = n - p - 1$ where $p $ is the number of predictors (explanatory variables) in the model.
\end{itemize}

\pause

\item Hypothesis testing for a slope: $H_0: \beta_1 = 0$; $H_A: \beta_1 \ne 0$ \\
\begin{itemize}
\item $T_{n-2} = \frac{b_1 - 0}{SE_{b_1}}$
\item p-value = P(observing a slope at least as different from 0 as the one observed if in fact there is no relationship between $x$ and $y$
\end{itemize}

\pause

\item Confidence intervals for a slope: 
\begin{itemize}
\item $b_1 \pm T^\star_{n-2} SE_{b_1}$
\end{itemize}

\end{itemize}

\end{frame}

%%%%%%%%%%%%%%%%%%%%%%%%%%%%%%%%%%%%

\subsection{Conditions for regression}
\label{mi3}

%%%%%%%%%%%%%%%%%%%%%%%%%%%%%%%%%%%%

\begin{frame}
\frametitle{Conditions for regression}

\begin{itemize}

\item Linearity $\rightarrow$ randomly scattered residuals around 0 in the residuals plot -- important regardless of doing inference

\pause

\item Nearly normally distributed residuals $\rightarrow$ histogram or normal probability plot of residuals -- important for inference

\pause

\item Constant variability of residuals (\hl{homoscedasticity}) $\rightarrow$ no fan shape in the residuals plot -- important for inference

\pause

\item Independence of residuals (and hence observations) $\rightarrow$ depends on data collection method, often violated for time-series data --  important for inference

\end{itemize}

\end{frame}

%%%%%%%%%%%%%%%%%%%%%%%%%%%%%%%%%%%%

\begin{frame}
\frametitle{Checking conditions}

\twocol{0.5}{0.5}
{
\clicker{What condition is this linear model obviously and definitely violating?}
\begin{enumerate}[(a)]
\item Linear relationship
\item Non-normal residuals
\item \solnMult{Constant variability}
\item Independence of observations
\end{enumerate}
}
{
\begin{center}
\includegraphics[width=\textwidth]{figures/problems/heteroscedastic}
\end{center}
}

\end{frame}

%%%%%%%%%%%%%%%%%%%%%%%%%%%%%%%%%%%

\begin{frame}
\frametitle{Checking conditions}

\twocol{0.5}{0.5}
{
\clicker{What condition is this linear model obviously and definitely violating?}
\begin{enumerate}[(a)]
\item \solnMult{Linear relationship}
\item Non-normal residuals
\item Constant variability
\item Independence of observations
\end{enumerate}
}
{
\begin{center}
\includegraphics[width=\textwidth]{figures/problems/nonlinear}
\end{center}
}

\end{frame}

%%%%%%%%%%%%%%%%%%%%%%%%%%%%%%%%%%%

\subsection{Type of outlier determines how it should be handled}
\label{mi4}

%%%%%%%%%%%%%%%%%%%%%%%%%%%%%%%%%%%%

\begin{frame}
\frametitle{Type of outlier determines how it should be handled}

\twocol{0.55}{0.5}{
\begin{itemize}
\item \pink{Leverage} point is away from the cloud of points horizontally, does not necessarily change the slope
\item \green{Influential} point changes the \underline{slope} (most likely also has high leverage) -- run the regression with and without that point to determine
\end{itemize}
}
{
\begin{center}
\includegraphics[width=\textwidth]{figures/outlier_sketch}
\end{center}
}

\pause

\begin{adjustwidth}{-0.25cm}{0cm}

\begin{itemize}
\item  \hl{Outlier} is an unusual point without these special characteristics (this one likely affects the intercept only)
\item If clusters (groups of points) are apparent in the data, it might be worthwhile to model the groups separately.
\end{itemize}

\end{adjustwidth}

\end{frame}

%%%%%%%%%%%%%%%%%%%%%%%%%%%%%%%%%%%%

\begin{frame}
\frametitle{}

\vfill

\app{6.2 Linear regression}{See course website for details}

\vfill

\end{frame}

%%%%%%%%%%%%%%%%%%%%%%%%%%%%%%%%%%%

\section{Summary}

%%%%%%%%%%%%%%%%%%%%%%%%%%%%%%%%%%%

\begin{frame}
\frametitle{Summary of main ideas}

\vfill

\begin{enumerate}

\item \nameref{mi1}

\item \nameref{mi2}

\item \nameref{mi3}

\item \nameref{mi4}

\end{enumerate}

\vfill

\end{frame}

%%%%%%%%%%%%%%%%%%%%%%%%%%%%%%%%%%

\end{document}